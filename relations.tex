\chapter{Relações}
Em capítulos anteriores, discutimos proposições que lidavam com a relação entre objetos matemáticos. Muitas vezes na matemática, e mesmo no contexto em que estamos inseridos, estamos interessados em definir e estudar relações entre objetos distintos. Por exemplo, podemos estar interessados em certas própriedades sobre a relação \textit{é vais velho que}, entre seres vivos, e diremos que essa é uma relação \textit{irreflexiva}, \textit{transitiva}, ou ainda, uma relação de \textit{ordem estrita}.

Nesse capítulo discutimos exatamente essas noções, e definimos certos tipos de relaçõs mais comuns.

\section{Semântica das Relações}
Podemos abstrair a noção semântica de relação para um universo $R$ de tuplas de aridade definida, que contem a sequencia dos objetos relacionados. Por exemplo, considere que o elemento $a$ está relacionado a $b$ por uma relação $R$. Ppodemos denotar $aRb$, ou $R(a,b)$ significando que o par $(a,b)\in R$, está definido como existente no universo daquela relação. Como se pode esperar, a relação pode ter qualquer aridade necessária, e relacionar objetos de tipos distintos.

Considere, a partir disso, o universo de objetos $u = \{Ana, Bia, Cid\}$, e a relação \textit{conhece}, definida por $U\times U \supseteq A = \{(Ana, Bia), (Bia, Cid)\}$. Podemos dizer que \textit{Bia conhece Ana}?

\section{Definições}
Iniciamos a seção propondo uma série de definições de relações extremamente úteis a partir das próximas subseções.

\subsection{Relações de Ordem}
Definições usando $\leq$ e o $<$... Podemos definir uma relação estrita por uma parcial, e uma parcial por uma estrita... etc. Algum exemplo não numérico!

\subsection{Relações de Equivalência}
Descrevemos as propriedades que definem esse tipo de relação, e damos exemplos. Mostramos as notações $a\sim b $, $a\equiv b$. Sugiro as relações de equivalencia "paralelo a", "modulo n", "mesma idade".

\subsection{Equivalencia e Igualdade}
Apenas discutimos brevemente como e porque Equivalencia e Igualdade são animais completamente diferentes. Toma os exemplos acima pra discutir.

\section{Relações em Lean}

\section{Exercícios}
