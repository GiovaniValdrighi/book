\chapter{Relações}
Em capítulos anteriores, discutimos proposições que lidavam com a relação entre objetos matemáticos. Muitas vezes na matemática, e mesmo no contexto em que estamos inseridos, estamos interessados em definir e estudar relações entre objetos distintos. Por exemplo, podemos estar interessados em certas própriedades sobre a relação \textit{é vais velho que}, entre seres vivos, e diremos que essa é uma relação \textit{irreflexiva}, \textit{transitiva}, ou ainda, uma relação de \textit{ordem estrita}.

Nesse capítulo discutimos exatamente essas noções, e definimos certos tipos de relaçõs mais comuns.

\section{Conceito de Relações}

Dados dois conjuntos $A$ e $B$, uma relação de $A$ em $B$ é qualquer subconjunto de $A\times B$ (Produto  cartesiano).\\

\textit{Exemplo:} $A = \{1,2,3\} $ e $ B = \{4,5,6\}$

$R_{1} = \{(1,4),(2,5)\}$

$R_{2} = \{(1,4),(1,5),(2,4),(3,5)\}$\\

Para efeito de nomenclatura, considere, por exemplo, que o elemento $a \in A$ está relacionado a $b \in B$ por uma relação $R$, Podemos denotar $aRb$, ou $R(a,b)$ significando que o par $(a,b)\in R$, está definido como existente no universo daquela relação. Como se pode esperar, a relação pode ter qualquer aridade necessária, e relacionar objetos de tipos distintos.

O domínio de uma relação $R$ de $A$ em $B$ é o conjunto formado pelos elementos de $A$ que se relacionam com alguém de $B$, no caso da relação $R_{1}$, por exemplo, $Dom(R_{1}) = \{1,2\}$.

O contradomínio de uma relação $R$ de $A$ em $B$ é o conjunto de elementos de $B$ que podem se relacionar com elementos de $A$, ou seja, é o própio $B$, $Cd(R_{1}) = Cd(R_{2}) =  \{4,5,6\}$.

A imagem de uma relação $R$ de $A$ em $B$ é o conjunto de elementos de $B$ que são relacionados por alguém de $A$, no caso da relação $R_{2}$, por exemplo, $Im(R_{2}) = \{4,5\}$.

Considere, a partir disso, o universo de objetos $u = \{Ana, Bia, Cid\}$, e a relação \textit{conhece}, definida por $U\times U \supseteq A = \{(Ana, Bia), (Bia, Cid)\}$. Podemos dizer que \textit{Bia conhece Ana}?



%\section{Definições}
\theoremstyle{definition}
\newtheorem{definition}{Definição}[section]

%\theoremstyle{definition}
%\newtheorem{example}{Exemplo}[section]

\theoremstyle{plain}
\newtheorem{theorem}{Proposição}[section]

%\theoremstyle{plain}
%\newtheorem{corollary}{Corolário}[section]

Definimos uma série de tipos de relações importantes, frequentemente encontradas na literatura.
Note que muitas das definições se aplicam a relações conhecidas como \textit{maior que}, nos naturais, ou \textit{pertence} para conjuntos.

\section{Relações em apenas um conjunto}

Sendo $U$ um conjunto, trata-se de uma relação de $U$ em $U$.\\

\textit{Ex:} $\{1,2,3,4,5\} \rightarrow  \{1,2,3,4,5\}$

$R = \{(1,2),(2,3),(3,2),(4,5)\}$

\subsection{Propiedades:}
     \begin{enumerate}
        \item Relexiva : Em uma relação relexiva, cada elemento de $U$ relaciona-se com ele mesmo.
        
        $R$ é reflexiva $\Leftrightarrow (\forall x)(x  \in  U \rightarrow xRx)$ 
        
        \textit{Ex:} $\{1,2,3\} \rightarrow  \{1,2,3\}$
        
        $R = \{(1,1),(2,2),(3,3),(2,3),(1,3)\}$
        
        \item Simétrica : Em um relação simétrica, cada par de elementos de $U$ se relacionam mutuamente.
        
        $R$ é simétrica $\Leftrightarrow (\forall x)( \forall  y)(x \in U \wedge y \in U \wedge xRy  \rightarrow  yRx)$
        
        \textit{Ex:} $\{1,2,3\} \rightarrow  \{1,2,3\}$
        
        $R = \{(1,2),(2,1),(2,2)\}$
        
        
        \item Antissimétrica : Em uma relação antissimétrica, os elementos só se relacionam mutuamente se forem iguais.
        
        $R$ é Antissimétrica $\Leftrightarrow (\forall x)( \forall  y)(x \in U \wedge y \in U \wedge (xRy  \wedge yRx) \rightarrow x=y)$
        
        \textit{Ex:} $\{1,2,3\} \rightarrow  \{1,2,3\}$
        
        $R = \{(1,2),(2,2),(3,1)\}$
        
        \item Transitiva : Em uma relação transitiva, quando um elemento $x\in U$ de relaciona com  $y\in U$, e este $y$ se relaciona com $z \in U$, então $x$ se relaciona com $z$.
        
        $R$ é transitiva $\Leftrightarrow ( \forall x)( \forall  y)( \forall z)(x \in U \wedge y \in U \wedge z \in U \wedge   (xRy  \wedge yRz) \rightarrow xRz)$
        
        \textit{Ex:} $\{1,2,3\} \rightarrow  \{1,2,3\}$
        
        $R = \{(1,2),(2,3),(1,3)\}$
        
        
     \end{enumerate}

\section{Relações de Ordem}
Discutimos uma classe de relações binárias importantes: as chamadas relações de ordem, que é uma relação ao mesmo tempo, relexiva, antissimétrica e transitiva. Aqui, definimos relações \textit{parciais} ou \textit{estritas}.
Usaremos os símbolos $\leq$ e $<$ para nos referir a relações quaisquer entre elementos de alguma estrutura $A$, e os usamos infixados: $x \leq y$ ou $x < y$.

\begin{definition}
    \label{partial_order}
    Seja $\leq$ uma relação. Dizemos que $\leq$ é de \textit{ordem parcial} se respeita as seguintes propriedades:

    \begin{itemize}
        \item \textbf{reflexividade:} para todo $x \in A$, $x\leq x$.
        \item \textbf{transitividade:} para todo $x,y,z \in A$, se $x \leq y$, e $y \leq z$, então $x\leq z$.
        \item \textbf{antissimetria:} para todo $x,y \in A$, se $x\leq y$ e $y \leq x$, então $x=y$.
    \end{itemize}
\end{definition}

\noindent Note que se entendemos $\leq$ por um predicado binário, as definições acima são facilmente expressos em lógica de primeira ordem.
Exemplos desse tipo são: $\leq$ em $\mathbb{N}$, $\mathbb{Z}$, $\mathbb{Q}$, e $\mathbb{R}$ ou a inclusão $\supseteq$ para a classe dos conjuntos.

Há ainda uma classe especial de relações de ordem vistas a seguir:

\begin{definition}
    \label{partial_order_total}
    Dizemos que a relação de \textit{ordem parcial} $\leq$ é \textit{total} se:

    \begin{itemize}
        \item para todo $x, y \in A$, $x\leq y$ ou $y\leq x$.
    \end{itemize}
\end{definition}

\noindent Vale observar que nos exemplos anteriores, apenas $\leq$ é total.
De fato, tome $A = \mathcal{P}(\mathbb{N}) $, os conjuntos $x={3}$ e $y={5}$ subconjuntos de $A$; claramente não vale a completude de $\subseteq$ em $A$. % Ainda, as relações \textit{divide}, $x|y$, nos inteiros, e outras...

O que dizer, no entanto, das relações \textit{menor} ou \textit{pertence}? De fato, essas pertencem a classe a seguir, as chamadas relações de \textit{ordem estrita}:

\begin{definition}
    \label{estrict_order}
    Considere $<$ relação em um conjunto $A$. Dizemos que a relação é de \textit{ordem estrita} $\leq$ é \textit{total} se:

    \begin{itemize}
        \item \textbf{transitividade: } para todo $x,y,z \in A$, se $x<y$ e $y<z$ então $x<z$.
        \item \textbf{irreflexividade: } para todo $x\in A$, $x \nless x$.
    \end{itemize}
    Dizemos, ainda, que essa relação estrita é total em $A$ se:
    \begin{itemize}
        \item \textbf{tricotomia:} para todo $x,y \in A$, vale $x<y$, $x>y$ ou $x=y$.
    \end{itemize}
\end{definition}

\noindent Novamente, é facil ver como formalizar essas noções utilizando proposições em lógica de primiera ordem.

% uma relação estrita é assimétrica: prova.

A seguir, discutimos um resultado intuitivo que estabelece uma ligação importante entre as relações de ordem \textit{parciais} e \textit{estritas}:

\begin{theorem}
    \label{estrict_by_partial}
    Considere $\leq$ parcial em $A$. Podemos definir uma relação estrita $<$ em $A$, em que $x<y$ significa que $x\leq y$ e $x \neq y$. Ainda, se $\leq $ for total, então $<$ também será total.
\end{theorem}
\begin{theorem}
    \label{partial_by_estrict}
    Considere $<$ estrita em $A$. Podemos definir a relação de ordem parcial $\leq $ em $A$, em que $x\leq y$ significa que $x < y$ ou $x = y$. Ainda, se $<$ for total, então $\leq$ também será total.
\end{theorem}

\begin{proof}
    Exercício para o leitor!
\end{proof}

\section{Relações de Equivalência}
%Descrevemos as propriedades que definem esse tipo de relação, e damos exemplos. Mostramos as notações $a\sim b $, $a\equiv b$. Sugiro as relações de equivalencia "paralelo a", "modulo n", "mesma idade".


\subsection{Equivalencia e Igualdade}
Apenas discutimos brevemente como e porque Equivalencia e Igualdade são animais completamente diferentes. Toma os exemplos acima pra discutir.

\section{Relações em Lean}

\section{Exemplos}
Os exemplos abaixo serão eventuais exercícios.
Mas já são dados com as soluções. Alguns poderão ser retirados e postos no interior do capítulo para ser usados como exemplos.

\begin{enumerate}
\item Dada uma relação de ordem estrita \textit{R}, definimos \textit{R'}, parcial. Dê uma prova para os seguintes resultados sobre \textit{R'}:

\vspace{5mm}
\begin{lstlisting}
section
-- Inicio da Sessao

parameters {A : Type} {R : A → A → Prop}
parameter (irreflR : irreflexive R)
parameter (transR : transitive R)

local infix < := R

def R' (a b : A) : Prop := R a b ∨ a = b
local infix ≤ := R'

-- Reflexividade de R'
example (a : A) : a ≤ a :=
  have h2 : a = a, from rfl,
  show a ≤ a, from (or.inr h2)

-- Transitividade de R'
example {a b c : A} (h1 : a ≤ b) (h2 : b ≤ c): a ≤ c :=
  or.elim h1
    (assume s1 : a < b,
      or.elim h2
        (assume s2 : b < c,
          or.inl (transR s1 s2))
        (assume s2 : b = c,
          or.inl (eq.subst s2 s1)))
    (assume s1 : a = b,
      eq.subst s1.symm h2)

-- Antissimetria de R'
example {a b : A} (h1 : a ≤ b) (h2 : b ≤ a) : a = b :=
  or.elim h1
    (assume s1 : a < b,
      or.elim h2
        (assume s2 : b < a,
          have s3: a < a, from transR s1 s2,
          false.elim (irreflR a s3))
        (assume s2 : b = a, s2.symm))
    (assume s1 : a = b, s1)

-- Fim da Sessao
end
\end{lstlisting}
\vspace{5mm}

\item Dada uma relação \textit{R}, definimos uma relação \textit{S} supostamente transitiva. Dê uma prova para esse fato.

\vspace{5mm}
\begin{lstlisting}
section
parameters {A : Type} {R : A → A → Prop}
parameter (reflR : reflexive R)
parameter (transR : transitive R)

def S (a b : A) : Prop := R a b ∧ R b a

example : transitive S :=
  assume a b c,
  assume h1 : S a b,
  assume h2 : S b c,
  show S a c, from
    have l1 : R a b, from h1.left,
    have l2 : R b c, from h2.left,
    have r1 : R b a, from h1.right,
    have r2 : R c b, from h2.right,
    ⟨transR l1 l2, transR r2 r1⟩
end
\end{lstlisting}
\vspace{5mm}

\item Apenas um dos teoremas é verdadeiro. Defina qual o verdadeiro, e dê uma prova para a sua resposta.

\vspace{5mm}
\begin{lstlisting}
section
  parameters {A : Type} {a b c : A} {R : A → A → Prop}
  parameter (Rab : R a b)
  parameter (Rbc : R b c)
  parameter (nRac : ¬ R a c)

  -- R e parcial estrita
  theorem R_strict : irreflexive R ∧ transitive R :=
  sorry

  -- R nao e parcial estrita
  theorem R_not_strict : ¬(irreflexive R ∧ transitive R) :=
    assume h : irreflexive R ∧ transitive R,
    have h1 : transitive R, from h.right,
    have h2 : R a c, from h1 Rab Rbc,
    show false, from nRac h2
end
\end{lstlisting}
\vspace{5mm}

\item Prove o fato a seguir utilizando paradigma \textit{calc}. Tente o mesmo apenas através de \textit{tatic mode} e \textit{term mode}.

\vspace{5mm}
\begin{lstlisting}
open nat

-- utilizando modo calc
example : 1 ≤ 4 :=
calc
    1 ≤ 2 : le_succ 1
  ... ≤ 3 : le_succ 2
  ... ≤ 4 : le_succ 3

-- utilizando term mode
example : 1 ≤ 4 :=
  have h1 : 1 ≤ 2, from le_succ 1,
  have h2 : 2 ≤ 3, from le_succ 2,
  have h3 : 3 ≤ 4, from le_succ 3,

  le_trans h1 (le_trans h2 h3)

-- utilizando tatic mode
example : 1 ≤ 4 := sorry
\end{lstlisting}
\vspace{5mm}

\end{enumerate}
