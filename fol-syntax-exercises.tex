    \subsection{Exercícios}

    \begin{enumerate}
        \item Formalize as seguintes afirmações em Lógica de Primeira Ordem, utilizando os seguintes predicados e relações: $primo$, $par$, $impar$, $perfeito$ e $amigos$.
        \begin{enumerate}
            \item Todo número primo maior do que $2$ é ímpar.
            \item A soma de dois números pares é par.
            \item Existe um número perfeito menor do que $10$.
            \item $220$ e $284$ são números amigos.
        \end{enumerate}
    \end{enumerate}

    \textbf{Soluções:}

    \begin{enumerate}
        \item
        \begin{enumerate}
            \item $\forall x \ ((primo(x) \land x > 2) \rightarrow impar(x))$.
            \item $\forall x \forall y \ ((par(x) \land par(y)) \rightarrow par(x + y))$.
            \item $\exists x \ (perfeito(x) \land x < 10)$.
            \item $amigos(220, 284)$.
        \end{enumerate}
    \end{enumerate}