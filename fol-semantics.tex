\section{Semântica}
    
    A seção anterior abordou a sintaxe da lógica de primeira ordem, ou seja, apresentou os símbolos e estruturas das fórmulas deste sistema. A semântica, por outro lado, define o valor-verdade das fórmulas pela \textbf{interpretação} e \textbf{valoração}.
   
   Analisemos alguns exemplos deste sistema lógico.
   Sejam:
   \begin{itemize}
       \item $\mathbb{N}$ um domínio escolhido;
       \item $0, 1, 2, 3$ símbolos constantes;
       \item \textit{adicao} e \textit{sucessor} os símbolos de função que operam neste domínio;
       \item  \textit{par}, \textit{impar}, \textit{menorDoQue},\textit{menorIgualDoQue} os simbolos de predicado .
   \end{itemize}
    A expressão 
   \begin{center}
       $\forall x \ ( \ menorDoQue(0, x) \ )$
   \end{center}
   
   é falsa, uma vez que para $x$ assumindo o valor de $0$, a relação \textit{menorDoQue} não é válida. Já a expressão
   
   \begin{center}
       $\forall x \ ( \ menorIgualDoQue( 0, x) \ )$
   \end{center}
    
    é verdadeira no domínio $\mathbb{N}$, porém se o domínio de interesse fosse $\mathbb{Z}$ a expressão se tornaria falsa.
    
    Logo, o valor verdade de uma sentença depende de como os quantificadores, o domínio, funções, predicados e relações são interpretados. 
    Porém, há algumas fórmulas que assumem sempre valor verdadeiro independente de qual for a interpretação, análogo à tautologia da lógica proposicional:
    
    \begin{center}
        $\forall x \ ( \  par(x) \rightarrow par(x) \ )$
    \end{center}
    
    Sentenças assim são chamadas \textbf{válidas}.
    
    Analogamente à valoração em lógica proposicional, há o \textbf{modelo} em lógica de primeira ordem. Enquanto a valoração permitia que atribuíssemos valores-verdade ( V ou F ) à todas as fórmulas da lógica proposicional, a escolha de um modelo permite a atribuição de valores-verdade a todas as sentenças da lógica de primeira ordem.
    
    \subsection{Interpretações}
    
    Usamos anteriormente alguns símbolos para representar predicados e constantes. Alguns deles:
    
    \begin{center}
        $0$, $1$, $ par$, $maiorDoQue$
    \end{center}
    
    
    Estes símbolos são autodescritivos considerando o domínio $\mathbb{N}$, e se torna natural sua valoração. Agora, sejam os seguintes predicados no mesmo domínio:
    
    \begin{center}
        $ligeiro$, $contente$, $facil$
    \end{center}
    
    Quando o predicado $ligeiro$ é verdadeiro? \\
    Não conseguimos responder, uma vez que não foram dadas informações suficientes.
    Se $ligeiro$ são os números ímpares, então $2, 4, 6, ...$ são  $ligeiro$ e   $1, 3, 5, ...$ não são. Seja $contente$ os múltiplos de $3$. Então $6$ é $ligeiro$ e $contente$. Se $ligeiro$ fossem os números da sequência de Fibonacci, então $6$ seria $contente$, mas não seria $ligeiro$.
    Cada explicação dada ( ímpar, múltiplos de $3$, números da sequência de Fibonacci) são \textbf{interpretações}. E vemos que é necessário a interpretação para a valoração. 
    
    
    Assim como os predicados, podemos interpretar as funções, relações e constantes:
    
    \begin{itemize}
        \item A interpretação de um predicado unário $P$ é um conjunto de elementos do domínio os quais $P$ é verdadeiro. 
        \item Para uma relação $R$ com aridade $n$, a interpretação é o conjunto de todas as tuplas com $n$ elementos para as quais $R$ é verdadeiro.
        \item E por fim, a interpretaçãode de uma função $f$ com aridade $n$, é uma função que relaciona $n$ elementos do domínio a outro elemento também do domínio.\\
     \end{itemize}
     
     É importante ressaltar a diferença entre símbolo sintático e semântica do predicado, função, relação e constante. Veja que não faz sentido escrevermos a relação $sucessorDe(4,3)$, pois $sucessorDe$ é um símbolo sintático sem significado por si só. \\
     Outra distinção importante é entre os objetos dos domínio e os símbolos constantes. Se considerarmos o domínio U de todas as cores, conhecemos os objetos deste domínio, mas podemos escrever o símbolo constante $verde$ e podemos interpretá-lo com verde ou como rosa, objetos do domínio.
     De maneira análoga, podemos definir os símbolos $0$, $1$, $2$ e interpretálos como os objetos do domínio $0$, $1$ e $2$.
     
    \subsection{Verdade em modelos}
    
     
    Na lógica proposicional a valoração dizia quais elementos deveriam ser interpretados como falsos e quais como verdadeiros. Já na lógica de primeira ordem serão avaliados cada termo e em seguida a interpretação é aplicada na estrutura. Mais adiante veremos exemplos que deixarão a ideia mais clara.
    
    Suponhamos um domínio D, em linguagem de primeira ordem, e uma interpretação em D para cada símbolo da linguagem. Um \textbf{modelo} é esta estrutura formada por um domínio D e a interpretação relativa a este domínio. Um modelo fornece as informações necessárias para avaliarmos todas as sentenças da linguagem em verdadeiro ou falso.
    
    Vamos relembrar a diferença entre termo e sentença e descrever as respectivas interpretações:
    
    \begin{itemize}
        \item  \textit{Termos} representam objetos e não possuem valor verdade. São termos: $a + b, f(x), c$. E a sua interpretação é definida pelo próprio modelo e são elementos do domínio. Para interpretar $a + b$, verificamos a interpretação do modelo para cada termo $a$ e $b$ e em seguida aplicamos a interpretação de $+$ à estes termos. Da mesma forma, para $f(x)$, analizamos a interpretação do termo $x$, dada pelo modelo e plicamos a interpretação de $f$ ao termo $x$.
        
        \item \textit{Sentenças} são relações ou predicados que assumem valor verdade. Alguns exemplos: $ R(a,b), x + y < x, P(a)$. Para interpretar um predicado ou uma relação primeiro se interpreta os termos como objetos do domínio, e em seguida verificar se a interpretação do símbolo da relação é verdadeira para estes objetos do domínio.
        
        
    \end{itemize}
    
    Seja A uma sentença e $\mathbb{M}$ um modelo da linguagem de A. Por questão de praticidade, vamos adotar as notações $\mathbb{M} \vDash A$ (pode ser lido como \textit{modela}, \textit{satisfaz} ou \textit{valida}) para quando o modelo $\mathbb{M}$ avalia a sentença $A$ como verdadeira ($ \textbf{T}$) e $\mathbb{M} \not \vDash A$ para quando $\mathbb{M}$ avalia $A$ como falsa ($ \textbf{F}$).
    
    Suponhamos um modelo $\mathbb{M}$ com domínio $\mathbb{N}$ de uma linguagem com símbolo de relação $maiorDoQue$, função binária $soma$
    e símbolos constantes $a$ e $b$ interpretados como $5$ e $8$, e a relação e função interpretados como os proprios nomes sugerem. Então $\mathbb{M} \not \vDash  maiorDoQue(a,b)$, pois $5$ não é maior que $8$, mas $\mathbb{M} \vDash maiorDoQue(soma(a,b),b)$, já que $5+8=13$ que é maior que $8$.
    
    Sejam $A$ e $B$ expressões em uma linguagem com interpretação $\mathbb{M}$ e domínio $D$, escrevemos $\mathbb{M} \vDash A \land B$, ou seja, $\mathbb{M}$ valida $A \and B$ somente quando $\mathbb{M} \vDash A$ e $\mathbb{M} \vDash B$. E os conectivos $\lor$, $ \rightarrow$ e $\neg$ também funcionam de maneira análoga ao conjunto proposicional.
    
    Agora veremos como interpretar expressões com os quantificadores existencial e universal. 
    
    Vimos que $\exists x A$, significa que existe algum elemento no domínio tal que a expressão $A$ seja verdadeira, ou seja, quando "substituirmos" $x$ por este elemento, $A$ seria verdadeira. Sendo mais preciso, dizemos que $\mathbb{M} \vDash\exists x A$ quando existe um elemento $a$ no domínio de $\mathbb{M}$ , tal que quando interpretamos $x$ como $a$, temos $\mathbb{M} \vDash \exists x A$. Usando o exemplo anterior temos $\mathbb{M} \vDash \exists x (maiorDoQue(x,b))$, uma vez que podemos interpretar $x$ como $10$ e assim $\mathbb{M} \vDash maiorDoQue(x,b)$.
    Finalmente, para o quantificador universal, vimos que $\forall x A$, significa que $A$ é verdadeira para toda variável do domínio. Sendo mais preciso, dizemos $\mathbb{M} \vDash \forall x A$ quando para todo elemento do domínio  de $\mathbb{M}$, se interpretarmos x como $a$ então $\mathbb{M} \vDash A$.
    
    Lembre-se que uma sentença é uma fórmula sem variáveis livres. Assim, as regras apresentadas determina o valor verdade de qualquer sentença de um modelo.
    
    \subsection{Exemplos}
    To do
    
    \subsection{Validade e consequência lógica}
    
    
    \subsection{Correção e completude}
    To do
    
    
    \begin{table}[htb]
      \centering
      \begin{tabular}{|l|l|l|l|}
      \hline
      G          & G          & B     &b           \\ \hline
      b          & b          & G     &G           \\ \hline
      g          & B          & b     &b            \\ \hline
      G          & g          & G     &B            \\ \hline
      
      \end{tabular}
      \end{table}

