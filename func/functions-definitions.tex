\begin{definition}
    \label{def1}
    Seja um conjunto $A$. A função identidade de $A$ é a função
    $i_A : A \rightarrow A$ definida para todos os valores $x \in A$ tal que $i_A(x) = x$.
\end{definition}

\begin{definition}
    \label{def2}
    Sejam $f : X \rightarrow Y$ e $g : Y \rightarrow Z$ funções.
    Defina $k : X \rightarrow Z$ por $k(x) = g(f(x))$. A função $k$ é chamada de composição
    de $f$ e $g$ ou $f$ composta com $g$ e é escrita $g \circ f$. Desta forma, para cada elemento,
    primeiro  avaliamos $f(x) \in Y$ e depois avaliamos $g(f(x)) \in Z$.
\end{definition}

Podemos ver em Lean as definições de composição e de identidade. Note que estamos em um namespace
hidden para que não haja conflito de definições.

\begin{lstlisting}
namespace hidden
    variables {X Y Z : Type}

    def comp (f : Y → Z) (g : X → Y) : X → Z :=
    λx, f (g x)

    infixr  ` ∘ ` := comp

    #check @comp

    def id (x : X) : X := x

    #check @id

end hidden
\end{lstlisting}

\begin{definition}
    \label{def3}
    Considere $f,g : X \to Y$. Dizemos que essas funções são iguais,
    quando para todos os valores do domínio $X$, a correspondência no contradomínio $Y$ é a
    mesma. Em lógica simbólica, $\forall x (x \in X \to (f(x) = g(x)) \iff f = g $.
\end{definition}

Obseve que em termos de lógica formal de tipos, poderíamos reescrever como
$\forall x : X (f(x) = g(x)) \iff f = g$. Escrevemos essa equivalência entre lógica e funções
utilizando a extensionalidade das funções, muito semelhante à descrita no capítulo de
\textit{\hyperlink{chapter.5}{conjuntos}}. Por exemplo, se $f, g : \mathbb{R} \to \mathbb{R}$
definidas por $f(x) = x + 1$ e $g(x) = 1 + x$, então $f = g$, pois para cada valor de $x$, vale
a comutativade da soma.

Em lean, o comando \lstinline{funext} (de "function extensionality") prova a igualdade de funções.

\begin{lstlisting}
variables {X Y : Type}

example (f g : X → Y) (h : ∀ x, f x = g x) : f = g :=
    funext h
\end{lstlisting}

\begin{theorem}
    \label{prop1}
    Para todo $f: X \to Y$, $f \circ i_X = f$ e $i_Y \circ f = f$
\end{theorem}
\begin{proof}
    Seja $x$ um elemento qualquer de $X$. Então $(f \circ i_X)(x) = f(i_X(x)) = f(x)$
     e $(i_Y \circ f)(x) = i_Y(f(x)) = f(x)$, o que mostra a igualdade.
\end{proof}

No Lean, podemos mostrar essa proposição da seguinte forma. Para algumas funções, será necessário
escrevermos \lstinline{open function}.

\begin{lstlisting}
variables {X Y Z W : Type}

lemma left_id (f : X → Y) : id ∘ f = f := rfl

lemma right_id (f : X → Y) : f ∘ id = f := rfl

theorem comp.assoc (f : Z → W) (g : Y → Z) (h : X → Y) :
    (f ∘ g) ∘ h = f ∘ (g ∘ h) := rfl

\end{lstlisting}

\begin{definition}
    \label{def4}
    Suponha $f:X \to Y$ e $g : Y \to X$ satisfaz $g \circ f = i_X$. Assim,
    $g(f(x)) = i_X(x) = x$, para todos os elementos de $X$. Neste caso $g$ é dita inversa à esquerda
    de $f$ e $f$ é dita inversa à direita de $g$. Quando $g$ é inversa à direita e é inversa à esquerda
    de $f$, então $g$ é dita simplesmente inversa de $f$.
\end{definition}

\begin{example}
    \label{ex1}
    Defina $f,g : \mathbb{R} \to \mathbb{R}$ por $f(x) = x + 1$ e $g(x) = x - 1$. Então $g$
    é inversa à direita e à esquerda de $f$ e vice-versa.
\end{example}
\begin{example}
    \label{ex2}
    $\mathbb{R}^{+}$ denota os reais não negativos. Defina $f : \mathbb{R} \to \mathbb{R}^{+}$
    por $f(x) = x^2$ e defina $g : \mathbb{R}^2 \to \mathbb{R}$ por $g(x) = \sqrt{x}$. Então
    $f(g(x)) = (\sqrt{x})^2 = x$, para todo $x$ no domínio de $g$. Então $f$ é inversa à esquerda de $g$
    e $g$ inversa à direita de $f$. Por outro lado, $g(f(x)) = \sqrt{x^2} = |x|$. Logo $g$ não é inversa à
    esquerda de $f$ e $f$ não é inversa à direita de $g$.
\end{example}

\begin{theorem}
    \label{prop2}
    Suponha que $f : X \to Y $ tem inversa à esquerda, $h : Y \to X $ e uma inversa à direita, $k : Y \to X $.
    Então $h = k$.
\end{theorem}
\begin{proof}
    Seja $y \in Y$. $h(f(k(y))) = k(y)$, pois $h$ é uma inversa à esquerda de $f$.
    Por outro lado, $f(k(y)) = y$ e, portanto $h(f(k(y))) = h(y)$. Assim $k(y) = h(y) $ e as funções são iguais.
\end{proof}

Essa proposição pode também ser vista em Lean. Considerarei a prova utilizando táticas. O leitor pode estudar
aquela que sentir mais confortável. Note que neste exemplo, já foi necessário o \lstinline{namespace function},
visto que \lstinline{left_inverse} e \lstinline{right_inverse} já estão predefinidas.

\begin{lstlisting}
open function

variables {X Y Z: Type}

-- Term and Calc Mode
example (f: X → Y) (h: Y → X) (k: Y → X)
    (hf: left_inverse h f) (fk: right_inverse k f): h = k :=
    have H: ∀ ( x : Y ), h x = k x, from
        assume x,
        have h1: h (f (k x)) = k x , from calc
            h ( f (k x)) = k x: by apply hf,
        have h2: h (f (k x)) = h x, from calc
            h (f (k x)) = h x: by rw fk,
        show h x = k x, from eq.trans (eq.symm h2) h1,
    show h = k, from funext H

-- Tatics Mode
example (f: X → Y) (h: Y → X) (k: Y → X)
    (hf: left_inverse h f) (fk: right_inverse k f): h = k :=
    begin
        apply funext,
        assume x,
        have hf: h (f (k x)) = k x, by apply hf,
        rw ←hf,
        rw fk,
    end
\end{lstlisting}

\begin{theorem}
    \label{prop3}
    Seja $f: X \to Y$. Se a inversa de $f$ existe, então ela é única. Isto é, se $g_1, g_2: Y \to X$ são inversas
    de $f$, então $g_1 = g_2$
\end{theorem}
\begin{proof}
    Sabemos que $g_1$ é inversa à esquerda de $f$. Então,
    $$g_1(f(g_2(x))) = g_2(x), ~para~todos~os~valores~de~x.$$
    Também, $g_2$ é inversa de $f$. Então,
    $$g_1(f(g_2(x))) = g_1(x), ~para~todos~os~valores~de~x.$$ Logo $g_1 = g_2$.
\end{proof}

Quando a inversa de $f$ existe, então, podemos escrevê-la como $f^{-1}$. Dada a Definição \hyperlink{def4},
podemos afirmar que $(f^{-1})^{-1} = f$.

Observe que uma função pode possuir mais de uma função inversa à esquerda ou mais de uma função inversa
à direita. Ainda, quando uma função possui mais de uma função inversa à esquerda, ela não possui inversa
à direita. Se ela possuisse, ela seria igual a todas as funções inversas à esquerda pela Proposição \ref{prop2},
portanto elas seriam iguais, o que é uma contradição, por hipótese. O mesmo vale para inversas à direita.

\begin{theorem}
     \label{prop4}
     Seja $f: X \to Y $ e $g: Y \to Z$. Se $h: Y \to X$ e $k: Z \to Y$ são inversas à esqueda de $f$ e $g$,
     respectivamente, então $h \circ k$ é inversa à esquerda de $g \circ f$. O mesmo vale quando substituimos
     esquerda por direita na proposição.
\end{theorem}

\begin{proof}
 $(h \circ k)\circ(g \circ f)(x) = h(k(g(f(x)))) = h(f(x)) = x $. A demonstração quando substituimos a proposição
 de esquerda para direita é análoga e é deixada como exercício ao leitor.
\end{proof}

\begin{corollary}
    \label{cor1}
    Se $f: X \to Y$ e $g: Y \to Z$ possuem inversas, então $(f \circ g)^{-1}$ existe e
    $(f \circ g)^{-1} = f^{-1} \circ g^{-1}$.
\end{corollary}

\begin{proof}
    Segue diretamente da Proposição \ref{prop4} e Proposição \ref{prop2}.
\end{proof}