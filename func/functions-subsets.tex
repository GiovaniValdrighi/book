Nós podemos querer saber o comportamento de uma função em algum subconjunto
$A$ de  $X$. Por exemplo, podemos dizer que $f$ é injetiva em $A$ se para
todo $x_1$ e $x_2$ em A, $f(x_1) = f(x_2) $ implicar $x_1 = x_2 $.

\begin{definition}
    \label{def8}
    Se $f$ é função de $X$ e $Y$, dizemos que $f[A]$ denota a imagem de $f$ em
    $A$, definido por $f[A] = \{y \in Y | ~\exists x \in A, y = f(x)\}$.
\end{definition}

\begin{theorem}
    \label{prop7}
    Seja $f: X \to Y $ e $A$ um subconjunto de $X$. Então, para todo $x$ em
    $A$, $f(x)$ está em $f[A]$.
\end{theorem}

\begin{proof}
    Por definição, $f(x) \in f[A] $ se, e somente se, existe $x'$ em $A$ tal que
    $f(x') = f(x)$. Isto vale para  $x' = x $.
\end{proof}

\begin{theorem}
    \label{prop8}
    Seja $f: X \to Y$ e $g: Y \to Z$. Seja $A$ subconjunto de $X$. Então
    $$(g \circ f)[A] = g[f[A]]$$
\end{theorem}

\begin{proof}
    Seja $z \in (g \circ f)[A]$. Então, para algum $x \in A$, $z = (g \circ f)(x) = g(f(x))$.
    Pelo que acabamos de provar na Proposição \ref{prop8}, $f(x) \in f[A]$. Novamente, pelo
    que acabamos de provar, $g(f(x)) \in g[f[A]] $.

    Alternativamente, seja $z \in g[f[A]]$. Então, existe $y$ em $f[A]$ tal que
    $f(y) = z$. Como $y \in f[A]$, existe $x \in A$, tal que $f(x) = y$. Então
    $(g \circ f)(x) = g(f(x)) = g(y) = z$, então $z \in (g \circ f)[A] $
\end{proof}

Uma prova de que a composição de funções sobrejetivas é
sobrejetiva é a que descrevemos cima, pois $f: X \to Y$ é
sobrejetiva se, e somente se, $f[X] = Y$.

Nós podemos ver $f$ como uma função de $A$, um subconjunto de $X$ a $Y$, simplesmente
ignorando o comportamento de $f$ nos elementos fora de $A$.

\begin{definition}
    \label{def9}
    Denotamos $f \upharpoonright A$ como restrição de $f$ para $A$. Isto é, dadas $f: X \to Y$
    e $A \subseteq X$, $f \upharpoonright A : A \to Y$ é definida por $(f \upharpoonright A)(x) = x$,
    para todo $x$ em $A$.
\end{definition}

Agora, $f$ é injetiva em $A$ significa que a restrição de $f$ em $A$ é injetiva.

\begin{definition}[Pré-imagem]
    \label{def10}
    Se $f: X \to Y$ e $B \subseteq Y$, então a pré-imagem de $B$ em $f$, denotado por $f^{-1}[B]$ é
    definida por $f^{-1}[B] = \{x \in X | f(x) \in B\}$. Ou seja, é o conjunto de elementos de $X$ que
    são mapeados em $B$.
\end{definition}

Note que essa definição faz sentido mesmo que $f$ não tenha inversa, visto que dado $y \in B$ pode não
haver $x \in X$ com a propriedade de $f(x) \in B$, como podem ter vários. Se $f$ tem inversa ($f^{-1}$),
então, para todo $y$ em $B$, existe exatamente um elemento $x$ em $X$ com $f(x) \in B$. Neste caso,
dizemos que $f^{-1}[B]$ é a imagem de $B$ sobre $f^{-1}$ ou a pré-imagem de $B$ sobre $f$.

\begin{theorem}
    Seja $f: X \to Y$, $g: Y \to Z$ e $C \subseteq Z$. Então $$(g \circ f)^{-1}[C] = f^{-1}[g^{-1}[C]]$$.
\end{theorem}
\begin{proof}
    Para qualquer $y$, $y \in (g \circ f)^{-1}[C]$ se, e somente se, $g(f(y))$ está em $C$.
    Isto acontece se, e somente se, $f(y) \in g^{-1}[C]$, o que acontece se, e somente se $y \in f^{-1}[g^{-1}[C]]$.
\end{proof}

Seguem algumas propriedades sobre imagens e pré-imagens. Aqui, $f$ denota uma função
arbitrária de $X$ em $Y$. $A, A_1, A_2, ...$ denotam subconjuntos arbitrários de $X$,
e $B, B_1, B_2,...$ denotam subconjuntos arbitrários de $Y$.

\begin{enumerate}
    \item $A \subseteq f^{-1}[f[A]]$, e se $f$ é injetiva, $A = f^{-1}[f[A]]$.
    \item $f[f^{-1}[B]] \subseteq B$, e se $f$ é sobrejetiva, $B = f[f^{-1}[B]]$.
    \item Se $A_1 \subseteq A_2$, então $f[A_1] \subseteq f[A_2]$.
    \item Se $B_1 \subseteq B_2$, então $f^{-1}[B_1] \subseteq f^{-1}[B_2]$.
    \item $f[A_1 \cup A_2] = f[A_1] \cup f[A_2]$.
    \item $f^{-1}[B_1 \cup B_2] = f^{-1}[B_1] \cup f^{-1}[B_2]$.
    \item $f[A_1 \cap A_2] \subseteq f[A_1] \cap f[A_2]$ e, se $f$ é injetiva, $f[A_1 \cap A_2] = f[A_1] \cap f[A_2]$.
    \item $f^{-1}[B_1 \cap B_2] = f^{-1}[B_1] \cap f^{-1}[B_2]$.
    \item $f[A] \setminus f[B] \subseteq f[A\setminus B]$.
    \item $f^{-1}[A] \setminus f^{-1}[B] \subseteq f[A \setminus B]$.
    \item $f[A] \cap B = f[A \cap f^{-1}[B]]$.
    \item $f[A \cup f^{-1}[B]] \subset f[A] \cup B $.
    \item $A \cap f^{-1}[B] \subseteq f^{-1}[f[A] \cap B]$.
    \item $A \cup f^{-1}[B] \subseteq f^{-1}[f[A] \cup B]$.
\end{enumerate}

A partir de agora, vamos demonstrar algumas dessas proporições, ora
com descrições da demonstração e ora com provas em Lean. Tambémm sugerimos
que essas proposições sejam exercícios para a leitora ou o leitor.

\subsection{Demonstrações com linguagem natural}

\begin{theorem}[Item 7]
    \label{exerc1}
    Sejam $X$ e $Y$ conjuntos, $A_1, A_2 \in X$ e $f: X \to Y$.
\end{theorem}

\begin{proof}
    Se $y \in f[A_1 \cap A_2]$, temos que existe $x \in A_1 \cap A_2$, com $f(x) = y$.
    Nesse caso, $f(x) \in f[A_1]$, e $f(x) \in f[A_2]$, o que demonstra a primeira parte.

    Agora, suponha a injetividade de $f$. Suponha também que $y \in f[A_1] \cap f[A_2]$. Assim, existem
    $x_1 \in A_1$ e $x_2 \in A_2$, com as propriedades de $f(x_1) = y = f(x_2)$. Como a função é injetiva,
    $f(x_1) = f(x_2) \Rightarrow x_1 = x_2$. Assim $x_1 \in A_2$, que implica $x_1 \in A_1 \cap A_2$. Logo,
    $y \in f[A_1 \cap A_2]$.
\end{proof}

\begin{theorem}[Item 11]
    \label{exerc2}
    Sejam $X$ e $Y$ conjuntos, $f: X \to Y, A \subseteq X$ e $B \subseteq Y$. Então,
    $f[A] \cap B = f[A \cap f^{-1}[B]]$
\end{theorem}

\begin{proof}
    Suponha, inicialmente, que $y \in f[A] \cap B$. Então $y \in B$ e para algum $x \in A$, $f(x) = y$.
    Nesse sentido, $x \in f^{-1}[B]$. Assim, $x \in A \cap f^{-1}[B]$ e, portanto, $y \in f[A \cap f^{-1}[B]]$.

    Alternativamente, se $y \in f[A \cap f^{-1}[B]]$, existe $x \in A \cap f^{-1}[B]$, com $f(x) = y$.
    Daqui, concluímos que $y = f(x) \in f[A]$ e $y \in B$, pela definição de pré-imagem. Então $y \in f[A] \cap B$,
    como queríamos provar.
\end{proof}

\subsection{Demontrações em Lean}

Nós podemos utilizar variáveis limitadas para falar sobre o comportamento de funções
em conjuntos particulares.

\begin{lstlisting}
import data.set -- inclui o símbolo de subconjunto da imagem ''
open set function

variables {X Y : Type}
variables (A  : set X) (B : set Y)

def maps_to (f : X → Y) (A : set X) (B : set Y) :=
    ∀ x ∈ A, f x ∈ B

def inj_on (f : X → Y) (A : set X) :=
    ∀ (x₁ ∈ A) (x₂ ∈ A), f x₁ = f x₂ → x₁ = x₂

def surj_on (f : X → Y) (A : set X) (B : set Y) :=
    B ⊆ f '' A

\end{lstlisting}

A definição de \lstinline{maps_on} é a ideia de que a imagem de um subconjunto do
domínio $X$ está totamente inclusa em um subconjunto específico do contradomínio. As
definições de \lstinline{inj_on} e \lstinline{surj_on} são as definições usuais.

\begin{theorem}[Item 3]
\end{theorem}

\begin{theorem}[Item 9]
\end{theorem}

Observe que para o item 9, trataremos $X$ e $Y$ como conjuntos iguais. Caso não sejam,
$f[B]$ pode nem estar definida.

\begin{lstlisting}
import data.set
open set function

variables {X Y : Type}
variables (A B A₁ A₂ : set X)

theorem item3 (f: X → Y) : A₁ ⊆ A₂ → f '' A₁ ⊆ f '' A₂ :=
    assume h : A₁ ⊆ A₂,
    assume y,
    assume h₁ : y ∈ f '' A₁,
    have h₂ : ∃ x, x ∈ A₁ ∧ f(x) = y, from h₁,
    show y ∈ f '' A₂, from exists.elim h₂
        (assume (x' : X) (ha: x' ∈ A₁ ∧ f(x') = y ),
        have h₃ : x' ∈ A₂ ∧ f(x') = y, from and.intro (h ha.left)  ha.right,
        show y ∈ f '' A₂, from exists.intro x' h₃)

theorem item9 (f: X → X) : f '' A \ f '' B ⊆ f '' (A \ B) :=
begin
    intros y h,
    have h₁ : y ∈ f '' A, from mem_of_mem_diff h,
    have h₂ : ¬ (y ∈ f '' B), from not_mem_of_mem_diff h,
    apply exists.elim h₁,
    intros x h₃,
    apply exists.intro x,
    apply and.intro,
    apply mem_diff_of_mem,
        exact h₃.left,
        assume h₄ : x ∈ B,
        exact false.elim (h₂ (exists.intro x (and.intro h₄ h₃.right))),
        exact h₃.right
end

#check @mem_of_mem_diff
#check @not_mem_of_mem_diff
#check @mem_diff_of_mem

\end{lstlisting}

As funções \lstinline{mem_of_mem_diff, not_mem_of_mem_diff} e \lstinline{mem_diff_of_mem}
tem o objetivo de lidar com a diferença esua relação com a interseção. Note que usar táticas
auxilia o passo a passo, porém dificulta o posterior entendimento. Por isso, é recomendável,
nesses exemplos, utilizar alguma ferramenta para Lean. 
Essas funções e outras já apresentadas, encontram-se na biblioteca do Lean e grande parte delas está 
disponível quando você abre o namespace \lstinline{function}.

\begin{lstlisting}
open function 

#check @comp 
#check @has_left_inverse
-- A right_inverse tem duas definicoes para Lean. Apenas uma
-- esta no namespace function. Por isso e importante especificar. 
#check @function.right_inverse    
\end{lstlisting}