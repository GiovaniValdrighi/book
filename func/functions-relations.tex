Podemos ver que nem toda relação é uma função. Uma relação binária $R(x,y)$ em
$A$ e $B$ é um \textit{funcional} se para todo $x$ em $A$, existe um valor
único $y$ em $B$ tal que $R(x,y)$. Se $R$ é uma relação funcional, podemos
definir $f_R: X \to B$ fazendo com que $f_R(x)$ seja igual ao valor de $y$
único em $B$ em que a relação seja verdadeira. A contrapartida é verdadeira,
basta definir, para $f: X \to B$, $R_f(x,y)$, uma relação funcional. Essa
relação é conhecida como o \textit{grafo de f}. 

Podemos extrair dessa ideia que existe uma dualidade entre uma relação
funcional $R$ e uma função $f$. Inclusive, muitas vezes, as definições são
intercambiáveis. Em fundamentos tipo-teóricos, como a que vemos em Lean,
relações são frequentemente definidas como uma classe de funções, enquanto
para fundamentos teóricos em conjuntos definem função como uma relação
funcional. 

Podemos considerar funções $f(x,y)$ ou $g(x,y,z)$ que tomam múltiplos
argumentos. Por exemplo, $f(x,y) = x + y$. Em lean, podemos exemplificar da
seguinte forma: 

\begin{lstlisting}
variables X Y : Type 
variable f: X → X → X → Y 
variables x₁ x₂ x₃  : X

#check f x₁ x₂ x₃ -- tipo Y

def g (x y : ℤ): ℤ := x + y

example : g 1 1 = 2 := 
calc 
g 1 1 = 1 + 1 : by rw g 
    ... = 2 :  rfl
\end{lstlisting}

O número de argumentos de uma função é chamado de \textit{aridade}, que em
inglês é \textit{arity}. No exemplo acima $f$ tem aridade $3$, enquanto $g$
tem aridade $2$. Podemos pensar uma função com múltiplos argumentos como uma
função unária definida em um produto cartesiano. Nesse sentido, quando estamos
nos tratando de teoria dos tipos, é conveniente pensar que $f$ é uma função de
$X$ que retorna uma função $X \to X \to Y$. Desta maneira, $f(x_1, x_2, x_3)$
abrevia $((f(x_1))(x_2))(x_3)$.

\begin{example}[Teorema de Cantor]
    \label{cantor}
    Seja $f$ uma função que mapeia elementos do conjunto $A$ para o seu
    conjunto das partes $P(A)$. Então $f: A \to P(A)$ não é sobrejetiva. Esse
    teorema é um teorema fundamental sobre conjuntos que foi apresentada nesse
    capítulo devido às definições importantes de funções. A referência é no
    capítulo sobre \hyperlink{chapter.5}{Conjuntos}. A demonstração se dá por
    contradição e é realizada no Lean com o auxílio do 
    \lstinline{namespace classical}. 

    \begin{lstlisting}
    import data.set

    variables X Y: Type 
    
    def surjective {X: Type} {Y: Type} (f : X → Y) : Prop := ∀ y, ∃ x, f x = y
    
    theorem Cantor : ∀ (A: set X), ¬ ∃ (f: A → set A),  surjective f :=
    
        begin
            intro A,
            intro h,
            -- Utiliza a regra eliminação existe.
            cases h with f h1,
            -- Como f é sobrejetiva e esse é um subconjunto de A            
            have h2: ∃ x, f x = {t : A | ¬ (t ∈ f t)},  
                from h1 {t : A | ¬ (t ∈ f t)},
            -- Eliminação do existe, novamente. 
            cases h2 with x h3,
            -- classical.em A := A ∪ ¬A é uma tautologia. 
            apply or.elim (classical.em 
                          (x ∈ {t : A | ¬ (t ∈ f t)})),
                intro h4,
            -- Usa a proriedade do conjunto
                    have h5: ¬ (x ∈ f x), from h4,
                    rw (eq.symm h3) at h4, 
                    apply h5 h4,
                intro h4,
                    rw (eq.symm h3) at h4,
                    have h5: x ∈ {t : A | ¬ (t ∈ f t)}, from h4,
                    rw (eq.symm h3) at h5,
                    apply h4 h5,                     
        end   
    \end{lstlisting}
\end{example}

Note que uma relação binária $R$ de $X$ em $Y$ é um funcional se satisfaz a
seguinte proposição:
\begin{math}
    \forall x, \exists! y R(x,y) 
\end{math}
Um lógico poderia usar a notação \textit{iota}, descrita como $f(x) = \iota y
R(x,y)$. Se $y$ não ser necessariamente único, um lógico utiliza a notação de
\textit{Helbert epsilon} $f(x) = \epsilon y R(x,y)$ para escolher um valor de
$y$ que satisfaça $R(x,y)$.

\subsection{Representação de Funções}

Apesar de não ser o objetivo nesse capítulo, considere a seguinte definição: 

\begin{definition}
    \label{represent}
    A grosso modo, uma função total $f(x_1, ..., x_k)$ é dita
    \textit{representável}, se existe uma expressão na linguagem formal,
    digamos $A$, que usa variáveis $x_1, ..., x_k, x_{k+1}$ tal que para
    quaisquer números $m_1,...,m_k$ e $n$, temos $f(m_1,...,m_k) = n$ se, e
    somente se, podemos provar em nosso sistema formal $A(m_1,...,m_k,n)$ e
    não podemos provar $A(m_1,...,m_k,j)$ para qualquer outro número $j$.
\end{definition}

Essa definição é encontrada no livro \cite{functionsComputaveis}. Apesar de
estar em termos um pouco diferentes dos que propomos nesse livro, essa
definição permite, em um sistema consistente, afirmar que toda função
representável é computável. Isso responde parcialmente nossa questão da
primeira sessão. Entretanto, em livros mais avançados de lógica, esse conceito
é mais discutido. 