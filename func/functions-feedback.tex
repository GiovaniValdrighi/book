\begin{enumerate}
    \item b e c. No primeiro caso, a função não é definida para $f(0)$, pois
    esse não tem antecessor. No quarto caso, a função não é definida para
    $h(0)$. 
    \item Seguem as provas: 
    \begin{enumerate}
        \item Sejam $x_1 \in X$ e $x_2 \in X$, com $f(x_1) = f(x_2)$. Como $g$
       é uma função bem definida, $g(f(x_1)) = (g \circ f)(x_1) = (g \circ
       f)(x_2) = g(f(x_2))$. Pela injetividade da composição, $x_1 = x_2$ está
        provado. 
        \item Para essa questão, existem vários exemplos. Seja $f : \mathbb{N}
        \to \mathbb{R} $ definida por $f(n) = n$ e seja $g : \mathbb{R} \to
        \mathbb{R}$, $g(x) = x^2 $. Temos que $(g \circ f) : \mathbb{N} \to
        \mathbb{R}$ é definida por $g(f(n)) = n^2$. Note que essa função é
        injetiva, porém $g$ não é. 
        \item Assuma $y_1, y_2 \in Y$. Se $g(y_1) = g(y_2)$, pela
        sobrejetividade de $f$, existem $x_1, x_2 \in X$, tal que $f(x_1) =
        y_1$ e $f(x_2) = y_2$. Pela injetividade da composição, $x_1 = x_2$.
        Como $f$ é uma função bem definida, $y_1 = f(x_1) = f(x_2) = y_2$, o
        que demostra a injetividade de $g$.
        \item Em lean, 
\begin{lstlisting}
import data.set 
open function
open set 

variables {X Y Z: Type}

theorem e2a (f : X → Y) (g: Y → Z) (h: injective (g ∘ f)) : injective f := 
assume x₁ x₂ h₁,
have h₂ : g(f(x₁)) = g(f(x₂)), from eq.subst h₁ (eq.refl (g(f x₁))), 
show x₁ = x₂, from h h₂

theorem e2c (f : X → Y) (g: Y → Z) (h₁: injective (g ∘ f)) (h₂: surjective f) :
        injective g := 
assume y₁ y₂ g₁, 
have g₂ : ∃ x, f(x) = y₁, from h₂ y₁,
have g₃ : ∃ x, f(x) = y₂, from h₂ y₂,
show y₁ = y₂, from exists.elim g₂ 
    (assume x₁ i₁, 
    show y₁ = y₂, from exists.elim g₃ 
        (assume x₂ i₂, 
        have i₃ : g(f(x₁)) = g(y₁), from eq.subst i₁ (eq.refl (g(f x₁))),
        have i₄ : g(f(x₂)) = g(y₂), from eq.subst i₂ (eq.refl (g(f x₂))),
        have i₅ : x₁ = x₂, from h₁ (eq.trans i₃ (eq.trans g₁ (eq.symm i₄))), 
        have i₆ : f(x₁) = f(x₂), from eq.subst i₅ (eq.refl (f(x₁))),
        show y₁ = y₂, from eq.trans (eq.trans (eq.symm i₁) i₆) i₂))    
\end{lstlisting}
    \end{enumerate}
    \item Seja $f: X \to Y$ monótona. Assuma $x_1, x_2 \in X$ e $f(x_1) =
    f(x_2)$. Se $x_1 < x_2$ ou $x_1 > x_2$, temos um absurdo. Logo $x_1 =
    x_2$, o que mostra que $f$ é monótona. 

    Para lean, eu faço uso de uma função \lstinline{ne_of_lt}, que significa
    que dado uma relação de menor ou maior entre dois naturais, eles são
    diferentes. 

\begin{lstlisting}
open function 

variables {X Y Z: Type}

def strict_cresc (f: ℕ → ℕ) := 
    ∀ x1, ∀ x2, x1 < x2 → f x1 < f x2 

def strict_decresc (f: ℕ → ℕ) := 
    ∀ x1, ∀ x2, x1 < x2 → f x1 > f x2 

def monotonous (f: ℕ → ℕ) : Prop := 
    strict_cresc f ∨ strict_decresc f 

lemma inj_of_cresc (f: ℕ → ℕ) (h: strict_cresc f) : injective f := 
assume x1 x2 h1,
nat.lt_by_cases
    (assume : x1 < x2, 
    have h2: f x1 < f x2, from h x1 x2 this,
    show x1 = x2, from false.elim (ne_of_lt h2 h1))
    (assume : x1 = x2, this)
    (assume : x1 > x2, 
    have h2: f x2 < f x1, from h x2 x1 this, 
    show x1 = x2, from false.elim (ne_of_lt h2 (eq.symm h1)))

lemma inj_of_decresc (f: ℕ → ℕ) (h: strict_decresc f) : injective f := 
assume x1 x2 h1,
nat.lt_by_cases
    (assume : x1 < x2, 
    have h2: f x1 > f x2, from h x1 x2 this,
    show x1 = x2, from false.elim (ne_of_lt h2 (eq.symm h1)))
    (assume : x1 = x2, this)
    (assume : x1 > x2, 
    have h2: f x2 > f x1, from h x2 x1 this, 
    show x1 = x2, from false.elim (ne_of_lt h2 h1))

theorem inf_mon (f: ℕ → ℕ)(h: monotonous f):injective f := 
begin 
    cases h, 
    apply inj_of_cresc f h, 
    apply inj_of_decresc f h
end

#check @ne_of_lt   
\end{lstlisting}

\item Suponha $y \in \mathbb{N}$. Note que posso escrever $2\cdot y = n$. De
fato, $g(n) = \left \lceil \frac{2y}{2} \right \rceil = \left \lceil y \right
\rceil = y $, pois $y$ é natural. Assim $\exists n \in \mathbb{N}, g(n) = y$.
Assim, a função é sobrejetiva. 

\item Considere o item 5 Suponha $y \in f[A_1 \cup A_2]$ Assim, existe $x \in
A_1 \cup A_2 $, tal que $f(x) = y$. Se $x \in A_1$, $f(x) = y \in f[A_1]$ e
análogo se $x \in A_2$. Logo $y \in f[A_1] \cup f[A_2]$. Alternativamente, se
$y \in f[A_1] \cup f[A_2]$. Suponha $y \in f[A_1]$. Então existe $x \in A_1$,
tal que $f(x)= y$ e $x \in A_1 \cup A_2$ que implica $y \in f[A_1 \cup A_2]$.
Análogo para $y \in f[A_2]$.

\begin{lstlisting}
import data.set
open function set

variables {X Y : Type}
variable  f : X → Y
variables A1 A2 : set X

example : f '' (A1 ∪ A2) = f '' A1 ∪ f '' A2 :=
eq_of_subset_of_subsetimport data.set
open function set

variables {X Y : Type}
variable  f : X → Y
variables A1 A2 : set X

example : f '' (A1 ∪ A2) = f '' A1 ∪ f '' A2 :=
eq_of_subset_of_subset
    (assume y,
    assume h1 : y ∈ f '' (A1 ∪ A2),
    exists.elim h1
    (assume x h,
    have h2 : x ∈ A1 ∪ A2, from h.left,
    have h3 : f x = y, from h.right,
    or.elim h2
        (assume h4 : x ∈ A1,
        have h5 : y ∈ f '' A1, from ⟨x, h4, h3⟩,
        show y ∈ f '' A1 ∪ f '' A2, from or.inl h5)
        (assume h4 : x ∈ A2,
        have h5 : y ∈ f ''  A2, from ⟨x, h4, h3⟩,
        show y ∈ f '' A1 ∪ f '' A2, from or.inr h5)))
    (assume y,
    assume h2 : y ∈ f '' A1 ∪ f '' A2,
    or.elim h2
        (assume h3 : y ∈ f '' A1,
        exists.elim h3 
        (assume x h,
        have h4 : x ∈ A1, from h.left,
        have h5 : f x = y, from h.right,
        have h6 : x ∈ A1 ∪ A2, from or.inl h4,
        show y ∈ f '' (A1 ∪ A2), from ⟨x, h6, h5⟩))
        (assume h3 : y ∈ f '' A2,
        exists.elim h3 
        (assume x h,
        have h4 : x ∈ A2, from h.left,
        have h5 : f x = y, from h.right,
        have h6 : x ∈ A1 ∪ A2, from or.inr h4,
        show y ∈ f '' (A1 ∪ A2), from ⟨x, h6, h5⟩)))
\end{lstlisting}
\end{enumerate}