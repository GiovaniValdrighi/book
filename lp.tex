\chapter{Lógica Proposicional}
\section{Introdução}

\section{Regras de Inferência}
Quando nos comunicamos, é habitual utilizar diferentes estruturas de linguagem que nos aproximem do sentido que queremos dar a alguma sentença. Na lógica proposicional, esses padrões são também amplamente utilizados e importantes para a construção de fórmulas e serão melhor aprofundados a seguir.
%Sugestões
%Acho que poderíamos juntar a parte de semântica do capítulo 6 
%(na qual ele discute o que é algo ser verdadeiro ou falso e introduz as tabelas verdade) 
%com os capítulos anteriores (caps 2-5), 
%onde ele introduz as regras de inferência, a dedução natural e o Lean. 
%Então introduziríamos as regras, mostrando o seu equivalente na tabela verdade e também no Lean. 
%Depois dessa seção mostraríamos as árvores mais complexas de dedução natural tb dessa forma. 

\subsection{Implicação} 
A implicação é essencial para o condicionamento de sentenças. Se na lingua falada utilizamos a estrutura "Se ... então" para nos referirmos a acontecimentos que dependem de outros, na logica a ideia é muito semelhante.

Retornando aos primeiros problemas do capítulo, relembramos um dos problemas descritos: % lembrar de add os exemplos na introdução %
\begin{center}
\textbf{Se Ana viajar para o Chile, comprará pesos chilenos}

\textbf{Se a família real não tivesse vindo ao Brasil, então o território se desintegraria}
\end{center}

Apesar das duas sentenças terem a mesma estrutura "Se... então", podemos perceber que as duas tem suas diferenças. Em particular, chamamos a segunda proposição de implicação contrafactual, pois ela afirma como o mundo possivelmente seria, caso a realidade fosse diferentes do que ela realmente é. Esse assunto é discutido por filósofos há seculos e Spinoza e Saul Kripke são nomes de destaque nesses estudos. Entretanto, a lógica matemática se debruça mais especialmente na primeira sentença.

Dessa forma, tomando a primeira sentença como objeto de estudo, como podemos valorar essa implicação?
Para começar a analisar a valoração dessa implicação pensemos em A correspondendo o evento "Ana viaja para o Chile" e B ao evento "Ana compra pesos chilenos". Temos, então, dois casos e uma tabela verdade correpondente:

\begin{center}

Caso 1: Ana viaja para o Chile 

Caso 2: Ana não viaja para o Chile

\end{center}

\begin{table}[htb]
\centering
\begin{tabular}{|l|l|l|}
\hline

\textbf{A} & \textbf{B} & \textbf{A $\to$ B} \\ \hline
V          & V          & V                  \\ \hline
V          & F          & F                  \\ \hline
F          & V          & V                  \\ \hline
F          & F          & V                  \\ \hline

\end{tabular}
\end{table}

No primeiro caso, Ana viaja para o Chile e A é verdadeiro. Dessa forma, para a implicação ser verdadeira, B precisa ver também verdadeiro.

No segundo caso, Ana não viaja para o Chile e A é falso.
Dessa forma, nada podemos dizer sobre o evento B e qualquer valor atribuído a ele torna a implicação verdadeira. A esse tipo de afirmação, chamamos de \textbf{verdadeira por vacuidade}.\\

Vamos agora analisar a implicação contrária:
\begin{center}

\textbf{Se Ana comprar pesos chilenos, viajará para o Chile}

\end{center}
Ainda analisando o evento A como "Ana viaja para o Chile" e B como "Ana compra pesos chilenos", temos a seguinte tabela verdade:

\begin{table}[htb]
\centering
\begin{tabular}{|l|l|l|}
\hline

\textbf{B} & \textbf{A} & \textbf{B $\to$ A} \\ \hline
V          & V          & V                  \\ \hline
V          & F          & F                  \\ \hline
F          & V          & V                  \\ \hline
F          & F          & V                  \\ \hline

\end{tabular}
\end{table}

A partir da comparação das duas tabelas podemos perceber que a segunda e terceira linhas evidenciam que, apesar de A e B terem o mesmo valor nas duas tabelas, a implicação tem uma valoração diferente. Ou seja, A $\to$ B $\neq$ B $\to$ A. Esse resultado condiz com a nossa intuição pois uma vez que embora saibamos que se Ana viaja para o Chile, comprará pesos chilenos, uma vez que Ana comprou pesos chilenos, não podemos afirmar com certeza que ela viajará para o Chile. Quem sabe Ana seja uma colecionadora de moedas internacionais?\\

Além disso, a implicação é protagonista da regra chamada "regra da exclusão da implicação" ou \textit{Modus Ponens}("A maneira que afirma afirmando" ), ou seja:

\begin{center}

Se Ana viajar para o Chile, comprará pesos chilenos

Ana viajou para o Chile

Ana comprou pesos chilenos

\end{center}

Escrito em dedução natural como:

\begin{prooftree}
    \AxiomC{$A \rightarrow B$}
    \AxiomC{A}
    \BinaryInfC{B}
\end{prooftree}

E com seu correspondente no Lean, sendo:

\vspace{5mm}
\begin{lstlisting} 

section
variable h1 : A → B
variable h2 : A

example : B := h1 h2
end

\end{lstlisting}
\vspace{5mm}

Por outro lado, existe também a "regra de inclusão da implicação". Uma vez que temos uma variável A e conseguimos derivar uma variável B, dizemos que \textbf{A implica em B}. E utilizamos em nossas árvores de dedução natural de tal forma:

\begin{prooftree}
    \AxiomC{$A$}
    \noLine
    \UnaryInfC{$\vdots$}
    \noLine
    \UnaryInfC{$B$}
    \UnaryInfC{$A \rightarrow B$}
\end{prooftree}

Enquanto no Lean:

\vspace{5mm}
\begin{lstlisting} 

example : A → B :=
assume h : A,
show B, from sorry

\end{lstlisting}
\vspace{5mm}

\subsection{Se e somente se}

Já vimos anteriormente que $A \rightarrow B \neq B \rightarrow A$. Entretanto, em muitos casos na Matemática conseguimos chegar na igualdade dessas duas implicações e precisamos expressar a estrutura da linguagem falada "Se, e somente se". Dessa forma, faz-se uso do símbolo chamado de "bi-implicação" e representado por $\iff$ \\
Poderíamos também utilizar a formalização $A \rightarrow B \land B \rightarrow A$, mas por questões ligadas a praticidade da abreviação, é preferível o uso do novo símbolo apresentado.\\
Para entender melhor a interpretação dessa regra, usaremos o exemplo abaixo:

\begin{center}
    \textbf{Ana viajará para o Chile se, e somente se, comprar pesos chilenos}
\end{center}

Novamente construiremos a bi-implicação tratando o evento A como sendo "Ana viaja ao Chile" e B "Ana compra pesos chilenos". A tabela verdade então será:

\begin{table}[htb]
\centering
\begin{tabular}{|l|l|l|}
\hline

\textbf{A} & \textbf{B} & \textbf{A $\iff$ B} \\ \hline
V          & V          & V                  \\ \hline
V          & F          & F                  \\ \hline
F          & V          & F                  \\ \hline
F          & F          & V                  \\ \hline

\end{tabular}
\end{table}

Para melhor visualização dos resultados, vamos também mostrar uma tabela verdade que utiliza a definição da bi-implicação com o "e":

\begin{table}[htb]
\centering
\begin{tabular}{|l|l|l|l|l|}
\hline

\textbf{A} & \textbf{B} & \textbf{A $\iff$ B}& \textbf{B $\iff$ A} & \textbf{(B $\iff$ A) $\land$ (A $\iff$ B)} \\ \hline
V          & V          & V                  & V                   & V\\ \hline
V          & F          & F                  & V                   & F\\ \hline
F          & V          & F                  & V                   & F\\ \hline
F          & F          & V                  & V                   & V\\ \hline

\end{tabular}
\end{table}

Dessa forma, temos os casos:
\begin{center}

Caso 1: Ana viaja para o Chile 

Caso 2: Ana não viaja para o Chile

Caso 3: Ana compra pesos chilenos

Caso 4: Ana não compra pesos chilenos

\end{center}
Assim, com essa sentença sabemos que o caso 1 acontece se, e somente se o caso 3 acontece, e o caso 2 acontece se, e somente se o caso 4 também acontece.\\
Em dedução natural a regra da inclusão bi-implicação evidencia a necessidade de possuirmos duas implicações verdadeiras. Escrevemos a regra como:

\begin{prooftree}
    \AxiomC{$A$}
    \noLine
    \UnaryInfC{$\vdots$}
    \noLine
    \UnaryInfC{$B$}
    \AxiomC{$B$}
    \noLine
    \UnaryInfC{$\vdots$}
    \noLine
    \UnaryInfC{$A$}
    \BinaryInfC{$A \iff B$}
\end{prooftree}

No lean, temos o comando "iff.intro" que introduz o símbolo dado a verdade das duas implicações:
\vspace{5mm}
\begin{lstlisting} 

example : A ↔ B :=
iff.intro
  (assume h : A,
    show B, from sorry)
  (assume h : B,
    show A, from sorry)

\end{lstlisting}
\vspace{5mm}

Enquanto para a exclusão, a regra em dedução natural é muito semelhante a regra da exclusão da implicação:

\begin{prooftree}
    \AxiomC{$A \iff B$}
    \AxiomC{$A$}
    \BinaryInfC{$B$}
\end{prooftree}
\begin{prooftree}
    \AxiomC{$A \iff B$}
    \AxiomC{$B$}
    \BinaryInfC{$A$}
\end{prooftree}

E seu correspondente no lean é feito pelos comandos de eliminação a direita e a esquerda:

\vspace{5mm}
\begin{lstlisting} 

section
  variable h1 : A ↔ B
  variable h2 : A

  example : B := iff.elim_left h1 h2
end

section
  variable h1 : A ↔ B
  variable h2 : B

  example : A := iff.elim_right h1 h2
end

\end{lstlisting}
\vspace{5mm}


\subsection{Conjunção}
A regra da conjunção se refere ao uso ``e'' na linguagem informal, de forma que juntamos duas informações. Podemos ter frases como:
\begin{center}
\textbf{Santiago é a capital do Chile e o deserto do Atacama está localizado no Chile.}
\end{center}

Ao mesmo tempo, podemos utilizar o ``e'' para conectar duas informações que nem mesmo possuem relação entre si. Por exemplo, podemos dizer que:
\begin{center}
\textbf{Santiago é a capital do Chile e Bolsonaro é o presidente do Brasil.}
\end{center}
Quando utilizamos o ``e'', representado pelo símbolo $\land$ na lógica, o que de fato importa é estarmos unindo duas informações que são verdadeiras. Isso fica claro na tabela verdade abaixo, na qual é possível ver que quando $A$ ou quando $B$ são falsos, $A\land B$ é falso:

\begin{table}[htb]
\centering
\begin{tabular}{|l|l|l|}
\hline
\textbf{A} & \textbf{B} & \textbf{A $\land$ B} \\ \hline
V          & V          & V                  \\ \hline
V          & F          & F                  \\ \hline
F          & V          & F                  \\ \hline
F          & F          & F                  \\ \hline
\end{tabular}
\end{table}

Dessa forma, na dedução natural, podemos introduzir um ``e'', quando temos $A$ e também temos $B$. Assim, se $A$ e $B$ forem verdadeiros, $A \land B$ também vai ser. 

 \begin{prooftree}
     \AxiomC{A}
     \AxiomC{B}
     \BinaryInfC{$A \land B$}
\end{prooftree}

No Lean, representamos essa operação pela função $and.intro$, conforme o exemplo abaixo:
\vspace{5mm}
\begin{lstlisting} 
variables A B : Prop

example (h1 : A) (h2 : B) : A ∧ B :=
and.intro h1 h2
\end{lstlisting}
\vspace{5mm}

Seguindo esse raciocínio, é possível observar que sempre que temos $A \land B$, teremos $A$ e $B$ separadamente. Essa é a operação chamada de exclusão do ``e''.

Para excluir o $B$ de, por exemplo, $A \land B$, temos a chamada exclusão pela esquerda, conforme descrita abaixo:

 \begin{prooftree}
     \AxiomC{$A \land B$}
     \UnaryInfC{A}
\end{prooftree}

No Lean, essa operação pode ser feita utilizando a função $and.left$, conforme o código a seguir: 
\vspace{5mm}
\begin{lstlisting} 
variables A B : Prop

example (h1 : A ∧ B): A :=
and.left h1
\end{lstlisting}
\vspace{5mm}

Alternativamente, é possível realizar a mesma operação da seguinte forma:
\vspace{5mm}
\begin{lstlisting} 
variables A B : Prop

example (h1 : A ∧ B): A :=
h1.left
\end{lstlisting}
\vspace{5mm}
Para excluir o $A$, temos a chamada exclusão pela direita:

 \begin{prooftree}
     \AxiomC{$A \land B$}
     \UnaryInfC{B}
\end{prooftree}

No Lean, essa operação é realizada utilizando a função $and.right$, de maneira semelhante ao que foi descrito anteriormente para o $and.left$. 

\subsection{Disjunção}

A regra da disjunção se refere ao uso do ``ou''. Na linguagem informal, podemos utilizá-lo para expressar situações excludentes, como da seguinte forma:
\begin{center}
\textbf{Alexandre vai viajar para o Atacama ou Alexandre vai viajar para Santiago.}\\
\end{center}
Nesse caso, é possível que uma pessoa compreenda que somente uma das duas situações vai ocorrer, ou seja, que Alexandre somente vai para um dos dois lugares no Chile. Contudo, quando utilizamos o  ``ou'' na lógica, representado pelo símbolo $\lor$, também estamos levando em consideração situações nas quais ambas as proposições são verdadeiras. Dessa forma, conforme indicado na tabela verdade a seguir, a sentença formada pelo ``ou'' será verdadeira quando somente $A$ for verdadeiro, quando somente $B$ for verdadeiro ou quando ambos forem verdadeiros. 

\begin{table}[htb]
\centering
\begin{tabular}{|l|l|l|}
\hline
\textbf{A} & \textbf{B} & \textbf{A $\lor$ B} \\ \hline
V          & V          & V                 \\ \hline
V          & F          & V                 \\ \hline
F          & V          & V                 \\ \hline
F          & F          & F                 \\ \hline
\end{tabular}
\end{table}

Não só o ``ou'' formará uma sentença verdadeira quando uma ou ambas as situações forem verdadeiras, como também quando ambas as situações forem verdadeiras, mas não possuírem nenhuma relação entre si. Por exemplo, a seguinte sentença é verdadeira: 
\begin{center}
\textbf{Santiago é a capital do Chile ou Bolsonaro é o presidente do Brasil.}\\
\end{center}

É verdadeiro que Santiago é a capital do Chile e que Bolsonaro é o presidente do Brasil. Logo, toda a sentença é verdadeira, apesar de não soar de forma natural na linguagem informal.

Na dedução natural, como o ``ou'' é verdadeiro mesmo que somente um de seus elementos seja verdadeiro, introduzimos o $\lor$ tendo somente $A$ ou somente $B$. Logo, pode-se formar $A \lor B$ da seguinte forma:
\begin{prooftree}
     \AxiomC{A}
     \UnaryInfC{$A \lor B$}
\end{prooftree}

%adicionar explicacao de como o lean entende que o B ficara do lado direito do ou

No Lean, essa operação de introdução do ``ou'' é realizada utilizando a função $or.inl$. Essa função indica que queremos adicionar o $A$ do lado esquerdo do $\lor$ (por isso, ``or include left''). Dessa forma, teremos o seguinte código: 
\begin{lstlisting} 
variables A B : Prop

example (h1 : A): A ∨ B :=
or.inl h1
\end{lstlisting} 

Alternativamente, é possível realizar a introdução do $\lor$ a partir do $B$:
\begin{prooftree}
     \AxiomC{B}
     \UnaryInfC{$A \lor B$}
\end{prooftree}

Nesse caso, como o $B$ está sendo adicionado à direita do $\lor$ utilizamos a função $or.inr$ no Lean (ou seja,  ``or include right''). 

\begin{lstlisting} 
variables A B : Prop

example (h1 : B): A ∨ B :=
or.inr h1
\end{lstlisting} 

Para excluir o ``ou'' de $A \lor B$ é preciso estruturar uma árvore de dedução natural de forma que $A$ e $B$ resultem em um mesmo resultado $C$. Chegando nesse $C$, é possível substituir o $A \lor B$ pelo C. A seguinte árvore de dedução natural demonstra essa estrutura:

\begin{prooftree}
    \AxiomC{$A \lor B$}
    \AxiomC{$A$}
    \noLine
    \UnaryInfC{$\vdots$}
    \noLine
    \UnaryInfC{$C$}
    \AxiomC{$B$}
    \noLine
    \UnaryInfC{$\vdots$}
    \noLine
    \UnaryInfC{$C$}
    \TrinaryInfC{$C$}
\end{prooftree}
     
No Lean, para excluir o ``ou'', utiliza-se a função $or.elim$. Ao lado do $or.elim$ é necessário inserir a hipótese contendo o ``ou'',  que deseja-se eliminar. Em seguida, deve-se abrir dois parênteses. Esses dois parênteses serão equivalentes às duas colunas contendo, respectivamente, $A$ e $B$ na árvore de dedução natural acima. Dessa forma, um parênteses começa com o $assume$ de $A$ e o outro com o$assume$ de $B$. O objetivo é que a partir deles chegue-se à hipótese C.  O código abaixo mostra a estrutura de como ficaria uma operação de eliminação do ``ou''(o ``sorry'' nas provas abaixo representa a etapa em que se prova que a partir de $A$ é possível chegar em $C$) :

\begin{lstlisting}
variables A B C D : Prop

example (h1: A ∨ B): C :=
or.elim h1
(assume h2: A, sorry)
(assume h2: B, sorry)
\end{lstlisting}

Como exemplo mais concreto, sem o uso do ``sorry'', é possível observar a prova abaixo, na qual utiliza-se as hipóteses $A \rightarrow C$ e $B \rightarrow C$ para se chegar à C a partir de $A$ e $B$:

\begin{lstlisting} 
variables A B C D : Prop

example (h1 : A → C) (h2 : B → C) (h3: A ∨ B): C :=
or.elim h3
(assume h4: A,
show C, from h1 h4)
(assume h4: B, 
show C, from h2 h4)
\end{lstlisting} 


\subsection{Negação}
A negação de $A$ é  representada  em  símbolos por $\neg A $. 
Mostrar que $\neg A $ ocorre, em termos lógicos, é o mesmo que mostrar que $A $ leva a uma contradição. Em dedução natural, temos uma prova que tem a seguinte árvore, na qual $\bot$ é o símbolo para falso, contradição ou absurdo:
\begin{prooftree}
    \AxiomC{}
    \RightLabel{\scriptsize(1)}
    \UnaryInfC{A}
    \noLine
    \UnaryInfC{$\vdots$}
    \noLine
    \UnaryInfC{$\bot$}
    \RightLabel{\scriptsize(1) $\neg$ I}
    \UnaryInfC{$\neg A$}
\end{prooftree}

Esta é uma outra forma de raciocínio hipotético, similar a utilizada em "se ... então" . Começamos supondo $A$. Em seguida, continuamos a prova aplicando as regras  já apresentadas, representadas pelos três pontinhos na árvore de dedução natural, até chegarmos a uma contradição.

A pergunta a se fazer em seguida é: de que forma uma contradição aparece numa prova, ou seja, como a encontramos? Suponha, por exemplo, que ao construir uma prova temos uma hipótese $A$ que nos leva a concluir que uma outra hipotése $B$ ocorre ao mesmo tempo que $\neg B$ também ocorre, no entanto se temos $B$ e  $ \neg B $, então temos uma contradição.
Em dedução natural, representamos essa ideia por:

\begin{prooftree}
    \AxiomC{$\neg B$}
    \AxiomC{B}
    \RightLabel{\scriptsize $\neg $ E}
    \BinaryInfC{$\bot$}
\end{prooftree}

A tabela verdade reforça o fato de que é impossível que uma proposição e sua negação sejam ambas verdadeiras ao mesmo tempo.

\begin{table}[htb]
\centering
\begin{tabular}{|l|l|}
\hline
\textbf{A} & \textbf{$\neg A$}  \\ \hline
V          & F                            \\ \hline
F          & V                            \\ \hline
\end{tabular}
\end{table}

Ao utilizarmos o Lean, as regras de inferência de eliminação e introdução da negação podem ser obtidas através dos comandos a seguir: 

\begin{lstlisting} 
variables p q : Prop
-- show false, from ...

example (h1 : P → Q) (h2 : ¬Q) : ¬P :=
assume hp : P,
    show false, from h2 (h1 hp)
\end{lstlisting} 

Note que a hipótese que contém a negação vem primeiro após o \verb|from| e \textit{Q} é resultado do uso da regra de eliminação da implicação, além disso o símbolo $\neg Q$ é escrito como \verb| \not Q|. Na biblioteca padrão do Lean, $ \neg P$ é na verdade uma abreviação de $ P \rightarrow false $, ou seja, o fato de que $P$ implica uma contradição.

Uma vez apresentado o símbolo para \textit{falso}, a seguir apresentamos o símbolo para \textit{verdadeiro}. No entanto,  \textit{verdadeiro} ou \textit{true} não possui regra de eliminação, apenas uma regra de introdução:
 \verb|true.intro: true|, às vezes abreviado como \verb|trivial: true|. Em outras palavras, verdadeiro é simplesmente verdadeiro e tem uma prova canônica, trivial.
 
\begin{prooftree}
    \AxiomC{}
    \UnaryInfC{$\top $}
\end{prooftree}

\subsection{Prova por contradição}

Existe um estilo de fazer matemática conhecido como "matemática construtivista" que nega a equivalência de $\neg \neg A$ e $ A$. De modo que uma demonstração de que algo é verdadeiro deveria fornecer evidências explícitas de que uma declaração é falsa, em vez de evidências de que ela não pode ser falsa. Uma prova construtivista é aquela que realmente lhe diz como encontrar o objeto que se afirma existir. 

A prova por contradição é informalmente utilizada para se referir a duas regras diferentes de inferência. Sendo elas:
\begin{itemize}
    \item Para provar "$\neg  P$"  é suficiente assumir "$P $" e derivar uma contradição;
    \item Para provar $ P$, basta assumir "não $ P$"  e derivar uma contradição.
\end{itemize}
\bigbreak
É importante que esteja bastante clara a diferença entre as duas regras. No primeiro argumento, uma negação é introduzida na conclusão, enquanto que no segundo, ela é eliminada da hipótese. A regra de inferência que conclui com a sentença positiva "$ P$" é chamada de redução ao absurdo (em latim, \textit{reductio ad absurdum}). De fato, a regra é equivalente ao princípio $ \neg \neg A \leftrightarrow A$. 

A primeira vista é difícil perceber claramente porque em uma delas temos uma prova construtivista e na outra uma que utiliza do raciocínio clássico. Na lógica clássica além de todas as regras de inferência já expostas neste livro, temos o princípio do terceiro excluído que diz que uma proposição $A $ ou é verdadeira ou é falsa, não existindo uma terceira opção. Assim, se é impossível uma proposição $P$ ser falsa, logo ela só pode ser verdadeira. 

Na linguagem natural, comumente tomamos a frase "Maria não não estava vestida de azul" como um modo redundante e agramatical de dizer que "Maria estava vestida de azul". Ao mesmo tempo que em "Eu não vi ninguém" as duas palavras negativas não se anulam. 

Em dedução natural, a redução ao absurdo é expressa do seguinte modo, onde a hipótese $\neg A$ é cancelada no final da inferência:

\begin{prooftree}
 \AxiomC{}
 \RightLabel{\scriptsize(1)}
 \UnaryInfC{$\neg A$}
 \noLine
 \UnaryInfC{$\vdots$}
 \noLine
 \UnaryInfC{$\bot $}
 \RightLabel{\scriptsize(1) R.A.A}
 \UnaryInfC{$A $}
\end{prooftree}

As regras de introdução e eliminação que vimos até agora no Lean são todas construtivaistas, ou seja, refletem um entendimento computacional dos conectivos lógicos com base na correspondência proposições como tipos. Para utilizar os princípios da lógica clássica, você deve abrir o modo clássico com o comando \verb|open classical| no início do seu arquivo ou em qualquer lugar antes de usá-lo.

\begin{lstlisting} 
open classical

variable P : Prop
#check em P
\end{lstlisting} 
Na linha 4, o comando \verb|#check| permite verificar se expressões que escrevemos estão bem formadas, e também pedir ao sistema para nos dizer que tipo de objeto eles denotam. Para o exemplo acima obtemos como output:
\begin{verbatim}
4:0: information: check result
em P : P ∨ ¬P
\end{verbatim}

No exemplo abaixo provamos $P$ a partir de $\neg \neg P$ utilizando redução ao absurdo. Note que não é preciso escrever explicitamente ao final o que queremos provar uma vez que o Lean, no caso da prova por contradição, infere que o resultado é ($\neg \neg P \to P $ ) a partir das hipóteses assumidas. 

\begin{lstlisting}
open classical

variable P : Prop

example (h : ¬¬ P) : P :=
by_contradiction
  (assume h1 : ¬ P,
    show false, from h h1)

\end{lstlisting}

%Verificar se a árvore está correta
\begin{prooftree}
 \AxiomC{}
 \RightLabel{\scriptsize(1)}
 \UnaryInfC{$\neg P$}
 \AxiomC{$\neg \neg P$}
 \BinaryInfC{$\bot $}
 \RightLabel{\scriptsize(1)}
 \UnaryInfC{$\neg \neg P \to P$}
\end{prooftree}



\section{Exemplos adicionais de Dedução Natural}

Nesta seção serão apresentados exemplos adicionais de provas de dedução natural e seus equivalentes no Lean. Também será explicado como construir essas árvores do zero, partindo somente do  que se quer provar.  


\textbf{1. Prova de A $\rightarrow$ C a partir de A $\rightarrow$ B e B $\rightarrow$ C:}

\begin{prooftree}
    \AxiomC{}
    \RightLabel{\scriptsize(1)}
    \UnaryInfC{A}
                               \AxiomC{$A \rightarrow B$}
                   \BinaryInfC{B}
                                        \AxiomC{$B \rightarrow C$}
                                 \BinaryInfC{C}
                                 \RightLabel{\scriptsize(1)}
                                 \UnaryInfC{$A \rightarrow C$}
\end{prooftree}

Como construir essa prova? Primeiro, começamos pelo que se quer provar: $A \rightarrow C$. Podemos escrever isso na última linha, já que é onde queremos chegar. Como a prova é de uma implicação, sabemos que na linha logo acima dessa vamos chegar no $C$:
\begin{prooftree}
\AxiomC{C}
\UnaryInfC{$A \rightarrow C $}
\end{prooftree}

Agora podemos considerar as hipóteses. Como a prova é de $A\rightarrow C$, vamos utilizar o $A$ em algum momento na árvore. Além disso, pelo enunciado, sabemos que vamos também utilizar o  $A \rightarrow B$ e $B \rightarrow C$. Então teremos uma estrutura razoavelmente parecida com essa: 

\begin{prooftree}
    \AxiomC{}
    \RightLabel{\scriptsize(1)}
    \UnaryInfC{A}
        \noLine
        \UnaryInfC{$\vdots$}
    \AxiomC{$A \rightarrow B$}
        \noLine
        \UnaryInfC{$\vdots$}
    \AxiomC{$B \rightarrow C$}
        \noLine
        \UnaryInfC{$\vdots$}
    \TrinaryInfC{$C$}
    \RightLabel{\scriptsize(1)}
    \UnaryInfC{$A \rightarrow C $}
\end{prooftree}
     
A partir disso, é possível observar com mais facilidade quais regras de dedução natural podem ser aplicadas ao problema. No caso, observamos que temos $A$ e $A\rightarrow B$. Logo, é possível obter $B$. 

\begin{prooftree}
    \AxiomC{}
    \RightLabel{\scriptsize(1)}
    \UnaryInfC{A}
    \AxiomC{$A \rightarrow B$}
    \BinaryInfC{$B$}
    \AxiomC{$B \rightarrow C$}
        \noLine
        \UnaryInfC{$\vdots$}
    \BinaryInfC{$C$}
    \RightLabel{\scriptsize(1)}
    \UnaryInfC{$A \rightarrow C $}
\end{prooftree}
     
Por fim, obtemos $C$ a partir de $B$ e de $B \rightarrow C$, formando a árvore apresentada de início. 

No Lean, essa prova poderia ser feita da seguinte forma: 
\begin{lstlisting}
variables A B C: Prop
example (h1: A → B) (h2: B → C): A → C :=
assume h3: A,
have h4: B, from h1 h3,
h2 h4
\end{lstlisting}

Para escrevê-la, adicionamos ao lado de $example$ as hipóteses que não vão ser descartadas. No caso, como o enunciado diz que vamos utilizar $A\rightarrow B$ e $B\rightarrow C$ para realizar a prova, adicionamos os dois depois de $example$. Em seguida, adicionamos $:$ e o que queremos provar, seguido de $:=$. 

Para o teor da prova, devemos considerar que a ordem no qual escrevemos no Lean importa. Logo, devemos seguir razoavelmente a ordem da árvore de dedução natural. Na árvore desse problema, é possível observar que a prova começa com o $A$ (do $A \rightarrow C$). Como essa é uma hipótese que vai ser descartada, escrevemos ela com o $assume$. Como a hipótese $A \rightarrow B$ já está escrita no $example$, podemos utilizá-la para encontrar $B$. Nesse caso, utilizamos o $have$ para atribuir o nome de uma variável ao $B$. Isso permite uma organização maior e evita que, em provas longas, tenha que se repetir muitas vezes como encontrar uma determinada variável. Contudo, não é necessário utilizar o $have$. Nesse caso, por exemplo, somente utilizamos o $B$ uma vez, para aplicá-lo ao $B \rightarrow C$. Logo, poderíamos ter escrito a prova da seguinte forma:

\begin{lstlisting}
variables A B C: Prop
example (h1: A → B) (h2: B → C): A → C :=
assume h3: A,
h2 (h1 h3)
\end{lstlisting}

A prova acima também pode ser intuitivamente pensada como (A $\rightarrow$ B) $\land$ (B $\rightarrow$ C) $\rightarrow$ (A $\rightarrow$ C). Nesse caso, teria-se a seguinte árvore:

\begin{prooftree}
    \AxiomC{}
    \RightLabel{\scriptsize(1)}
    \UnaryInfC{A}
               \AxiomC{}
               \RightLabel{\scriptsize(2)}
               \UnaryInfC{$(A \rightarrow B) \land (B \rightarrow C)$}
               \UnaryInfC{$A \rightarrow B$}
        \BinaryInfC{B}
                                           \AxiomC{}
                                           \RightLabel{\scriptsize(2)}
                                           \UnaryInfC{$(A \rightarrow B) \land (B \rightarrow C)$}
                                           \UnaryInfC{$B \rightarrow C$}
                          \BinaryInfC{$C$}
                          \RightLabel{\scriptsize(1)}
                          \UnaryInfC{$A \rightarrow C$}
                          \RightLabel{\scriptsize(2)}
                          \UnaryInfC{$(A \rightarrow B) \land (B \rightarrow C) \rightarrow (A \rightarrow C)$}
\end{prooftree}

Para construir a árvore acima, poderia-se adotar essa estratégia: primeiro, começamos pelo que se quer provar ($(A \rightarrow B) \land (B \rightarrow C) \rightarrow (A \rightarrow C)$). Como vamos chegar em $A \rightarrow C$, podemos colocá-lo logo acima do que queremos provar. Ainda assim, continuamos com uma implicação, então podemos inserir o $C$ antes do $A \rightarrow C$. 

Como hipótese, teremos o o $A$ (de $A \rightarrow C$) e o $(A \rightarrow B) \land (B \rightarrow C)$ (vindo do resultado final: $(A \rightarrow B) \land (B \rightarrow C) \rightarrow (A \rightarrow C)$). Assim, teremos uma estrutura dessa forma: 
\begin{prooftree}
    \AxiomC{}
    \RightLabel{\scriptsize(1)}
    \UnaryInfC{A}
        \noLine
        \UnaryInfC{$\vdots$}
    \AxiomC{}
    \RightLabel{\scriptsize(2)}
    \UnaryInfC{$(A \rightarrow B) \land (B \rightarrow C)$}
        \noLine
        \UnaryInfC{$\vdots$}
    \BinaryInfC{$C$}
    \RightLabel{\scriptsize(1)}
    \UnaryInfC{$A \rightarrow C$}
    \RightLabel{\scriptsize(2)}
    \UnaryInfC{$(A \rightarrow B) \land (B \rightarrow C) \rightarrow (A \rightarrow C)$}
\end{prooftree}
     
Como sabemos que queremos chegar em um $C$, podemos utilizar o $B \rightarrow C$ para isso ($B \rightarrow C$ pode ser obtido a partir da operação de exclusão do $\land$ na hipótese de número 2) . Contudo, é necessário ter $B$ para realizar essa operação. O $B$ pode ser obtido a partir do $A$ e do $A \rightarrow B$. Logo, é possível construir a árvore. 

No Lean essa prova poderia ser feita da seguinte forma: 
\begin{lstlisting}
variables A B C: Prop
example: ((A → B) ∧ (B → C)) → (A → C) :=
assume h1: (A → B) ∧ (B → C),
assume h2: A,
have h3: B, from (and.left h1) h2,
(and.right h1) h3
\end{lstlisting}

Nas duas árvores de deduçao natural acima, podemos observar a utilização de números em determinadas hipóteses. Utilizamos esses para evidenciar onde descartamos as hipóteses marcadas. 

\textbf{2. Prova de $Q\land S$ a partir de $(P\land Q)\land R$ e $S \land T$:}

\begin{prooftree}
    \AxiomC{$(P \land Q) \land S$}
    \UnaryInfC{$P \land Q$}
    \UnaryInfC{Q}
                                      \AxiomC{$S \land T$}
                                      \UnaryInfC{S}
                      \BinaryInfC{$Q \land S$}
\end{prooftree}
Nesse caso, não descartamos nenhuma hipótese pois assumimos $(P\land Q)\land R$ e $S \land T$ como verdade e apenas derivamos a prova.

Para construir essa prova, partimos também de onde queremos chegar: $Q \land S$. Para formar um $\land$, precisamos de $Q$ e de $S$ separadamente. Logo, podemos escrevê-los acima do $Q \land S$. Pelo enunciado, sabemos que vamos usar como hipótese: $(P\land Q)\land R$ e $S \land T$. Assim, teremos a seguinte estrutura: 

\begin{prooftree}
    \AxiomC{$(P \land Q) \land S$}
     \noLine
    \UnaryInfC{$\vdots$}
     \noLine
    \UnaryInfC{$Q$}
    \AxiomC{$S \land T$}
     \noLine
    \UnaryInfC{$\vdots$}
     \noLine
    \UnaryInfC{S}
    \BinaryInfC{$Q \land S$}
\end{prooftree}

É possível obter o $S$ a partir de qualquer uma das hipóteses. O $Q$ só é possível obter a partir de $(P \land Q) \land S$. Logo, a partir de uma série de operações de exclusão do $\land$, constrói-se a parte restante da prova. 

No Lean, essa prova poderia ser feita da seguinte forma: 

\begin{lstlisting}
variables P Q R S T: Prop
example (h1: (P ∧ Q) ∧ R) (h2: S ∧ T): Q ∧ S :=
have h3: Q, from and.right (and.left h1),
have h4: S, from and.left h2,
and.intro h3 h4
\end{lstlisting}

\textbf{3. Prova de $(A \rightarrow (B \rightarrow C)) \rightarrow (A \land B \rightarrow C)$} :
\begin{prooftree}
    \AxiomC{}
    \RightLabel{\scriptsize(2)}
    \UnaryInfC{$A \rightarrow (B \rightarrow C)$}
                                               \AxiomC{}
                                               \RightLabel{\scriptsize(1)}
                                               \UnaryInfC{$A \land B$}
                                               \UnaryInfC{A}
                        \BinaryInfC{$B \rightarrow C$}
                                                                          \AxiomC{}
                                                                          \RightLabel{\scriptsize(1)}
                                                                          \UnaryInfC{$A \land B$}
                                                                          \UnaryInfC{B}
                                                      \BinaryInfC{C}
                                                      \RightLabel{\scriptsize(1)}
                                                      \UnaryInfC{$A \land B \rightarrow C$}
                                                      \RightLabel{\scriptsize(2)}
                                                      \UnaryInfC{$(A \rightarrow (B \rightarrow C)) \rightarrow (A \land B \rightarrow C) $}
\end{prooftree}

%essa prova é bem parecida com as outras, preciso explicar?

\textbf{4. Prova de $A \land B \iff B \land A$:}
\begin{prooftree}
    \AxiomC{}
    \RightLabel{\scriptsize(1)}
    \UnaryInfC{$A \land B$}
    \UnaryInfC{B}
                              \AxiomC{}
                              \RightLabel{\scriptsize(1)}
                              \UnaryInfC{$A\land B$}
                              \UnaryInfC{A}
             \BinaryInfC{$B \land A$}
                                                         \AxiomC{}
                                                         \RightLabel{\scriptsize(2)}
                                                         \UnaryInfC{$B \land A $}
                                                         \UnaryInfC{A}
                                                                                    \AxiomC{}
                                                                                    \RightLabel{\scriptsize(2)}
                                                                                    \UnaryInfC{$B \land A$}
                                                                                    \UnaryInfC{B}
                                                                     \BinaryInfC{$A \land B$}
                                                                     \RightLabel{\scriptsize(1,2)}
                                      \BinaryInfC{$A \land B \iff B \land A$}
\end{prooftree}

Para construir essa prova, partimos do $A \land B \iff B \land A$ ao final da prova. Para obtê-lo, precisamos partir de $B \land A$ e chegar no $A \land B$ e também partir de  $A \land B$ e chegar em $B \land A$. Sabendo que chegaremos nos dois, podemos escrevê-los na linha acima de  $A \land B \iff B \land A$. Sabendo que partiremos também dos dois, podemos escrevê-los como hipóteses. Teremos, assim, uma estrutura como essa: 
\begin{prooftree}
    \AxiomC{}
    \RightLabel{\scriptsize(1)}
    \UnaryInfC{$A \land B$}
     \noLine
    \UnaryInfC{$\vdots$}
     \noLine
    \UnaryInfC{$B \land A$}
    \AxiomC{}
    \RightLabel{\scriptsize(2)}
    \UnaryInfC{$B \land A$}
     \noLine
    \UnaryInfC{$\vdots$}
     \noLine
    \UnaryInfC{$A \land B$}
    \RightLabel{\scriptsize(1,2)}
    \BinaryInfC{$A \land B \iff B \land A$}
\end{prooftree}

Para formar $A \land B$ e $B \land A$ precisamos de $A$ e $B$ separadamente. Estes podem ser obtidos a partir das hipóteses. Logo, repetindo as hipóteses e realizando a exclusão do $\land$, formamos a árvore final.

No Lean, essa prova poderia ser construída da seguinte forma:
\begin{lstlisting}
variables A B C: Prop
example: A ∧ B ↔ B ∧ A :=
iff.intro 
    (assume h1: A ∧ B,
    and.intro (and.right h1) (and.left h1))
    (assume h3: B ∧ A,
    and.intro (and.right h3) (and.left h3))

\end{lstlisting}

\textbf{5. Prova de $A \land (B \lor C) \rightarrow (A \land B) \lor (A \land C)$:}

\begin{prooftree}

\AxiomC{}                   
\RightLabel{\scriptsize(2)} 
\UnaryInfC{$A \land (B \lor C)$}                
\UnaryInfC{$B \lor C$}

\AxiomC{}
\RightLabel{\scriptsize(2)} 
\UnaryInfC{$A \land (B \lor C)$}
\UnaryInfC{A}
\AxiomC{}
\RightLabel{\scriptsize(1)} 
\UnaryInfC{B}
\BinaryInfC{$A \land B$}
\UnaryInfC{$(A \land B) \lor (A \land C)$}

\AxiomC{}
\RightLabel{\scriptsize(2)} 
\UnaryInfC{$A \land (B \lor C)$}
\UnaryInfC{A}
\AxiomC{}
\RightLabel{\scriptsize(1)} 
\UnaryInfC{C}
\BinaryInfC{$A \land C$}
\UnaryInfC{$(A \land B ) \lor (A \land C)$}

            \RightLabel{\scriptsize(1)} 
            \TrinaryInfC{$ (A \land B) \lor (A \land C)$}
            \RightLabel{\scriptsize(2)} 
           \UnaryInfC{$A \land (B \lor C) \rightarrow (A \land B) \lor (A \land C)$}
\end{prooftree}

Para escrever essa prova, podemos partir, como nas outras, do objetivo final: $A \land (B \lor C) \rightarrow (A \land B) \lor (A \land C)$. Como estamos provando uma implicação, vamos chegar em $(A \land B) \lor (A \land C)$ na linha anterior. Observamos também que $A \land (B \lor C)$ será uma hipótese. Teremos, então, uma estrutura como essa:

\begin{prooftree}
 \AxiomC{}
\RightLabel{\scriptsize(2)} 
\UnaryInfC{$A \land (B \lor C)$}
     \noLine
    \UnaryInfC{$\vdots$}
    \noLine
    \UnaryInfC{$(A \land B) \lor (A \land C)$}
    \RightLabel{\scriptsize(2)}
    \UnaryInfC{$A \land (B \lor C) \rightarrow (A \land B) \lor (A \land C)$}
\end{prooftree}

Como proceder a partir dessa estrutura? Para formar o $(A \land B) \lor (A \land C)$ será necessário o $A \land B$ isoladamente ou o $A \land C$ (a partir de qualquer um dos dois é possível realizar a introdução do $\lor$ e assim chegar em $(A \land B) \lor (A \land C)$). Independente de qual dos dois serão utilizados, nota-se que, para formar o ``e'', será necessário: o $A$ e também o $B$ ou o $C$. O $A$ pode ser facilmente obtido  a partir da hipótese. Para obter $B$ ou $C$ isoladamente, será necessário realizar a exclusão do $B \lor C$ contido no ``e'' da hipótese. Nota-se que para realizar a exclusão do ``ou'', é necessário considerar $B$ e $C$ como hipótese. Assim, podemos formar tanto $A \land B$, quanto $A \land C$. Dessa forma, é possível chegar no resultado final descrito acima. 

No Lean, essa prova poderia ser construída da seguinte forma:
\begin{lstlisting}
variables A B C: Prop
example: (A ∧ (B ∨ C)) → ((A ∧ B) ∨ (A ∧ C)) :=
assume h1: A ∧ (B ∨ C),
have h2: B ∨ C, from and.right h1,
have h4: A, from and.left h1,
or.elim h2
    (assume h3: B, 
    or.inl (and.intro h4 h3))
    (assume h3: C,
    or.inr (and.intro h4 h3))
\end{lstlisting}

\textbf{6. Prova do Princípio da Casa dos Pombos (PHP-3):}
    O princípio da casa dos pombos afirma que se $n$ pombos devem ser postos em $m$ casas, e se $n > m$, então pelo menos uma casa irá conter mais de um pombo. Apesar desse princípio  generalizado ser muito amplo para ser provado por árvores dedutivas, conseguimos provar o caso de 3 pombos e 2 casas de pombo utilizando árvores dedutivas.
    %tentaremos por a árvore?%

No Lean, essa prova poderia ser construída da seguinte forma: 
\begin{lstlisting}
variables P11 P12 P21 P22 P31 P32 : Prop

example: ((P11 ∨ P12) ∧ (P21 ∨ P22)) ∧ (P31 ∨ P32) → (P22 ∧ P32) ∨ ((P11 ∧ P31) ∨ ((P12 ∧ P22) ∨ ((P11 ∧ P21) ∨ ((P12 ∧ P32) ∨ (P21 ∧ P31))))) :=

assume h: ((P11 ∨ P12) ∧ (P21 ∨ P22)) ∧ (P31 ∨ P32),
have ha: P11 ∨ P12, from and.left (and.left h),
or.elim ha
    (assume h1: P11, 
        have hb: P21 ∨ P22, from and.right (and.left h),
            or.elim hb
                (assume h2: P21, or.inr (or.inr (or.inr (or.inl (and.intro h1 h2)))))
                (assume h2: P22, 
                    have hc: P31 ∨ P32, from and.right h,
                        or.elim hc
                            (assume h3: P31, or.inr (or.inl (and.intro h1 h3)))
                            (assume h3: P32, or.inl (and.intro h2 h3))))
    (assume h1: P12, 
        have hb: P21 ∨ P22, from and.right (and.left h),
            or.elim hb
                (assume h2: P21, 
                    have hc: P31 ∨ P32, from and.right h,
                        or.elim hc
                            (assume h3: P31, or.inr (or.inr (or.inr (or.inr (or.inr (and.intro h2 h3))))))
                            (assume h3: P32, or.inr (or.inr (or.inr (or.inr (or.inl (and.intro h1 h3)))))))
                (assume h2: P22, or.inr (or.inr (or.inl (and.intro h1 h2)))))
\end{lstlisting}

%Exemplos adicionais que eu tenho em Lean (temos que decidir quais vamos colocar como exercício, quais como exemplo e quais vamos fazer a árvore de dedução natural -- acho que faltam árvores contendo negação e contradição):
\begin{lstlisting}
open classical
variables {A B C D E F P Q R : Prop}

example : A ∧ (A → B) → B :=
assume h: A ∧ (A → B),
have h2: A → B, from and.right h,
have h3: A, from and.left h,
h2 h3

---
example : A → ¬ (¬ A ∧ B) :=
assume h1: A,
assume h2: ¬ A ∧ B,
have h3: ¬A, from and.left h2,
false.elim (h3 h1)

---
example : ¬ (A ∧ B) → (A → ¬ B) :=
assume h1: ¬ (A ∧ B),
assume h2: A,
assume h3: B,
have h4: A ∧ B, from and.intro h2 h3,
false.elim (h1 h4)

---
example (h1 : A ∨ B) (h2 : A → C) (h3 : B → D) : C ∨ D :=
or.elim h1
    (assume h: A, show C ∨ D, from or.inl (h2 h))
    (assume h: B, show C ∨ D, from or.inr (h3 h))

---
example (h : ¬ A ∧ ¬ B) : ¬ (A ∨ B) :=
assume h1: A ∨ B,
or.elim h1
    (assume h2: A, false.elim ((and.left h) h2))
    (assume h2: B, false.elim((and.right h) h2))

---
example : ¬ (A ↔ ¬ A) :=
assume h1: A ↔ ¬ A,
have h2: ¬ A, from
	assume h4: A, 
	have h2: ¬  A, from iff.elim_left h1 h4,
	false.elim (h2 h4),
have h3: ¬ ¬ A, from 
	assume h4: ¬ A,
	have h3: A, from iff.elim_right h1 h4,
	false.elim (h4 h3),
false.elim (h3 h2)

---
example (h1 : A ∨ ¬ A) (h2: ¬ A → false) : A := 
or.elim h1
    (assume h: A, 
    show A, from h)
    (assume h: ¬ A, 
    show A, from false.elim (h2 h))

---

example (h: ¬ A ∨ ¬ B): ¬ (A ∧ B) :=
assume hab: A ∧ B,
show false, from
    or.elim h
    (assume h2: ¬ A ,
    show false, from h2 (hab.left))
    (assume h3: ¬ B,
    show false, from h3 (hab.right))

---

lemma stepA (h₁ : ¬ (A ∧ B)) (h₂ : A) : ¬ A ∨ ¬ B :=
have ¬ B, from (assume hk:B,
show false, from h₁ (and.intro h₂ hk)),
show ¬ A ∨ ¬ B, from or.inr this

lemma stepB (h₁ : ¬ (A ∧ B)) (h₂ : ¬ (¬ A ∨ ¬ B)) : false :=
have ¬ A, from
  assume : A,
  have ¬ A ∨ ¬ B, from stepA h₁ ‹A›,
  show false, from h₂ this,
show false, from (h₂ (or.inl this) )

theorem stepC (h : ¬ (A ∧ B)) : ¬ A ∨ ¬ B :=
by_contradiction
  (assume h' : ¬ (¬ A ∨ ¬ B),
    show false, from stepB h h')


---

example (h5: ¬P →(Q ∨ R)) (h6:¬ Q) (h7:¬ R) :P:=
by_contradiction(
assume hp: ¬ P,
have hqr: Q ∨ R , from h5 hp,
show false, from
    or.elim hqr
    (assume hi: Q,
    show false, from h6 hi)
    (assume hii: R,
    show false, from h7 hii))

---

example (h8:A → B):¬ A ∨ B:=
or.elim(em A)
    (assume he: A,
    show (¬ A ∨ B),from or.inr (h8 he) )
    (assume hi: ¬ A,
    show(¬ A ∨ B),from or.inl hi)

---

example :A → (A ∧ B) ∨ (A ∧ ¬ B):=
assume h9: A,
show (A ∧ B) ∨  (A ∧ ¬ B), from
or.elim(em B)
    (assume hr:B,
    show (A ∧ B) ∨  (A ∧ ¬ B), from or.inl(and.intro h9 hr) )
    (assume hw: ¬ B,
    show (A ∧ B) ∨  (A ∧ ¬ B), from or.inr(and.intro h9 hw))

---

example: ((A ∨ B) ∧ (C ∨ D)) ∧ (E ∨ F) →  (((((((((A ∧ E) ∧ C) ∨ ((F ∧ B) ∧ D)) ∨ ((A ∧ F) ∧ C)) ∨ ((A ∧ E) ∧ D)) ∨ ((A ∧ F) ∧ D)) ∨ ((B ∧ E) ∧ C)) ∨ ((B ∧ F) ∧ C)) ∨ ((B ∧ E) ∧ D)) :=
assume h: ((A ∨ B) ∧ (C ∨ D)) ∧ (E ∨ F),
have ha: A ∨ B, from and.left (and.left h),
or.elim ha
    (assume h1: A, have hb: C ∨ D, from and.right (and.left h),
    or.elim hb
        (assume h2: C, have hc: E ∨ F, from and.right h,
        or.elim hc
            (assume h3: E, or.inl (or.inl (or.inl (or.inl (or.inl (or.inl (or.inl (and.intro (and.intro h1 h3) h2))))))))
            (assume h3: F, or.inl (or.inl (or.inl (or.inl (or.inl (or.inr (and.intro (and.intro h1 h3) h2))))))))
        (assume h2: D, have hc: E ∨ F, from and.right h,
        or.elim hc
            (assume h3: E, or.inl (or.inl (or.inl (or.inl (or.inr (and.intro (and.intro h1 h3) h2))))))
            (assume h3: F, or.inl (or.inl (or.inl (or.inr (and.intro (and.intro h1 h3) h2)))))))
    (assume h1: B, have hb: C ∨ D, from and.right (and.left h),
    or.elim hb
        (assume h2: C, have hc: E ∨ F, from and.right h,
        or.elim hc
            (assume h3: E, or.inl (or.inl (or.inr (and.intro (and.intro h1 h3) h2))))
            (assume h3: F, or.inl (or.inr (and.intro (and.intro h1 h3) h2))))
        (assume h2: D, have hc: E ∨ F, from and.right h,
        or.elim hc
            (assume h3: E, or.inr (and.intro (and.intro h1 h3) h2))
            (assume h3: F, or.inl (or.inl (or.inl (or.inl (or.inl (or.inl (or.inr (and.intro (and.intro h3 h1) h2))))))))))


---

lemma step1 (h₁ : ¬ (A ∧ B)) (h₂ : A) : ¬ A ∨ ¬ B :=
have ¬ B, from (assume hk:B,
show false, from h₁ (and.intro h₂ hk)),
show ¬ A ∨ ¬ B, from or.inr this

lemma step2 (h₁ : ¬ (A ∧ B)) (h₂ : ¬ (¬ A ∨ ¬ B)) : false :=
have ¬ A, from
  assume : A,
  have ¬ A ∨ ¬ B, from step1 h₁ ‹A›,
  show false, from h₂ this,
show false, from (h₂ (or.inl this) )

theorem step3 (h : ¬ (A ∧ B)) : ¬ A ∨ ¬ B :=
by_contradiction
  (assume h' : ¬ (¬ A ∨ ¬ B),
    show false, from step2 h h')

example (hl : ¬ B → ¬ A) : A → B :=
assume h10: A, show B,from
by_contradiction(
    assume hnb: ¬ B,
    show false, from (hl hnb) h10)

---
example (hp : A → B) : ¬ A ∨ B :=
show ¬ A ∨ B, from
or.elim (em A)
    (assume ha1: A,
    show ¬A ∨ B, from or.inr (hp ha1) )
    (assume ha2: ¬ A,
    show ¬A ∨ B, from or.inl ha2)
\end{lstlisting}

\begin{lstlisting}

---vestidos:
variables AA AB AP MA MB MP CA CB CP : Prop 
variable h1: AA ∨ (AB ∨ AP)
variable h2: MA ∨ (MB ∨ MP)
variable h3: CA ∨ CB ∨ CP
variable h4: AA → AB
variable h5: CA → ¬ AB
variable h6: AB → MB
variable h7: CB → ¬ MB
variable h8: AP → CB
variable h9: CP → ¬ CB
variable h10: ¬ AB
variable h11: (AA → (¬ AB ∧ ¬ AP)) ∧ (AB → (¬ AA ∧ ¬ AP)) ∧ (AP → (¬ AB ∧ ¬ AA))
variable h12: (MA → (¬ MB ∧ ¬ MP)) ∧ (MB → (¬ MA ∧ ¬ MP)) ∧ (MP → (¬ MB ∧ ¬ MA))
variable h13: (CA → (¬ CB ∧ ¬ CP)) ∧ (CB → (¬ CA ∧ ¬ CP)) ∧ (CP → (¬ CB ∧ ¬ CA))
variable h14: (AA → (¬ MA ∧ ¬ CA)) ∧ (AB → (¬ MB ∧ ¬ CB)) ∧ (AP → (¬ MP ∧ ¬ CP))
variable h15: (MA → (¬ AA ∧ ¬ CA)) ∧ ((MB → (¬ AB ∧ ¬ CB)) ∧ (MP → (¬ AP ∧ ¬ CP)))
variable h16: (CA → (¬ AA ∧ ¬ MA)) ∧ (CB → (¬ AB ∧ ¬ MB)) ∧ (CP → (¬ AP ∧ ¬ MP))

example: ((AP ∧ CB) ∧ MA) :=

have h17: AP, from 
or.elim h1
    (assume h18: AA, false.elim (h10 (h4 h18)))
    (assume h18: AB ∨ AP,
        or.elim h18
            (assume h19: AB, false.elim (h10 h19))
            (assume h19: AP, h19)),

have h20: CB, from 
(h8 h17),

have h24: (AP ∧ CB), from and.intro h17 h20,

have h21: MA, from 
or.elim h2
(assume h22: MA, h22)
(assume h22: MB ∨ MP,
or.elim h22
(assume h23: MB, false.elim ((and.right ((and.left (and.right h15)) h23)) (and.right h24)))
(assume h23: MP, false.elim ((and.left ((and.right (and.right h15)) h23)) (and.left h24)))),

and.intro h24 h21
\end{lstlisting}

\begin{lstlisting}

--hamiltonian path:

open classical 
variables {x11 x12 x13 x14 x21 x22 x23 x24 x31 x32 x33 x34 x41 x42 x43 x44 : Prop}

/-     grafo:               caminho:
     x1 ---- x2        x11 ∧ x22 ∧ x33 ∧ x44         
     |        |       e, naturalmente ¬xij para
     x4------x3       qualquer outro par i,j          
-/


--- REGRA 1 E 3 ---
lemma rule_1and3 (h1: x11 ∧ x22 ∧ x33 ∧ x44):
(((x11 ∨ x21 ∨ x31 ∨ x41)∧ 
(x44 ∨ x14 ∨ x24 ∨ x34))∧
(x33 ∨ x13 ∨ x23 ∨ x43))∧
(x22 ∨ x12 ∨ x32 ∨ x42)
  :=
  show 
  (((x11 ∨ x21 ∨ x31 ∨ x41)∧ 
  (x44 ∨ x14 ∨ x24 ∨ x34))∧
  (x33 ∨ x13 ∨ x23 ∨ x43))∧
  (x22 ∨ x12 ∨ x32 ∨ x42), 
  from and.intro 
        (and.intro 
            (and.intro 
                (or.inl h1.left)
                (or.inl h1.right.right.right))
            (or.inl h1.right.right.left))
        (or.inl h1.right.left)

--- REGRA 2 ---

--- para o nó 1 ---
lemma rule_2_node1 (h2: x11 ∧ ¬x21 ∧ ¬x31 ∧ ¬x41 ∧ x22 ∧ x33 ∧ x44):
  (((((¬ x11 ∨ ¬  x21)∧
      (¬ x11 ∨ ¬  x31))∧ 
      (¬ x11 ∨ ¬  x41))∧ 
      (¬ x21 ∨ ¬  x31))∧ 
      (¬ x21 ∨ ¬  x41))∧ 
      (¬ x31 ∨ ¬  x41)
  :=
  show(((((¬ x11 ∨ ¬  x21)∧
      (¬ x11 ∨ ¬  x31))∧ 
      (¬ x11 ∨ ¬  x41))∧ 
      (¬ x21 ∨ ¬  x31))∧ 
      (¬ x21 ∨ ¬  x41))∧ 
      (¬ x31 ∨ ¬  x41),
  from and.intro
          (and.intro
                    (and.intro
                          (and.intro
                            (and.intro
                              ( or.inr h2.right.left ) 
                              ( or.inr h2.right.right.left ))
                            (or.inr h2.right.right.right.left))
                          (or.inl h2.right.left))
                    (or.inl h2.right.left))
          (or.inl h2.right.right.left)

--- para o nó 2 ---
lemma rule_2_node2 (h2: x11 ∧ ¬x12 ∧ ¬x32 ∧ ¬x42 ∧ x22 ∧ x33 ∧ x44):
  (((((¬ x12 ∨ ¬  x22)∧
      (¬ x12 ∨ ¬  x32))∧ 
      (¬ x12 ∨ ¬  x42))∧ 
      (¬ x22 ∨ ¬  x32))∧ 
      (¬ x22 ∨ ¬  x42))∧ 
      (¬ x32 ∨ ¬  x42):=
  show
    (((((¬ x12 ∨ ¬  x22)∧
      (¬ x12 ∨ ¬  x32))∧ 
      (¬ x12 ∨ ¬  x42))∧ 
      (¬ x22 ∨ ¬  x32))∧ 
      (¬ x22 ∨ ¬  x42))∧ 
      (¬ x32 ∨ ¬  x42),
  from and.intro
          (and.intro
                    (and.intro
                          (and.intro
                            (and.intro
                              ( or.inl h2.right.left ) 
                              ( or.inl h2.right.left ))
                            (or.inl h2.right.left))
                          (or.inr h2.right.right.left))
                    (or.inr h2.right.right.right.left))
          (or.inl h2.right.right.left)

--- para o nó 3 ---
lemma rule_2_node3 (h2: x11 ∧ ¬x13 ∧ ¬x43 ∧ ¬x23 ∧ x22 ∧ x33 ∧ x44):
  (((((¬ x13 ∨ ¬  x23)∧
      (¬ x13 ∨ ¬  x33))∧ 
      (¬ x13 ∨ ¬  x43))∧ 
      (¬ x23 ∨ ¬  x33))∧ 
      (¬ x23 ∨ ¬  x43))∧ 
      (¬ x33 ∨ ¬  x43):=
  show
   (((((¬ x13 ∨ ¬  x23)∧
      (¬ x13 ∨ ¬  x33))∧ 
      (¬ x13 ∨ ¬  x43))∧ 
      (¬ x23 ∨ ¬  x33))∧ 
      (¬ x23 ∨ ¬  x43))∧ 
      (¬ x33 ∨ ¬  x43),
  from and.intro
          (and.intro
                    (and.intro
                          (and.intro
                            (and.intro
                              ( or.inl h2.right.left ) 
                              ( or.inl h2.right.left ))
                            (or.inl h2.right.left))
                          (or.inl h2.right.right.right.left))
                    (or.inl h2.right.right.right.left))
          (or.inr h2.right.right.left)

--- para o nó 4 ---
lemma rule_2_node4 (h2: x11 ∧ ¬x14 ∧ ¬x34 ∧ ¬x24 ∧ x22 ∧ x33 ∧ x44):
  (((((¬ x14 ∨ ¬  x24)∧
      (¬ x14 ∨ ¬  x34))∧ 
      (¬ x14 ∨ ¬  x44))∧ 
      (¬ x24 ∨ ¬  x34))∧ 
      (¬ x24 ∨ ¬  x44))∧ 
      (¬ x34 ∨ ¬  x44):=
  show
   (((((¬ x14 ∨ ¬  x24)∧
      (¬ x14 ∨ ¬  x34))∧ 
      (¬ x14 ∨ ¬  x44))∧ 
      (¬ x24 ∨ ¬  x34))∧ 
      (¬ x24 ∨ ¬  x44))∧ 
      (¬ x34 ∨ ¬  x44),
  from and.intro
          (and.intro
                    (and.intro
                          (and.intro
                            (and.intro
                              ( or.inl h2.right.left ) 
                              ( or.inl h2.right.left ))
                            (or.inl h2.right.left))
                          (or.inl h2.right.right.right.left))
                    (or.inl h2.right.right.right.left))
          (or.inl h2.right.right.left)

--- REGRA 4 ---

--- para a posição 1 ---
lemma rule_4_node1 (h2: ¬x12 ∧ ¬x13 ∧ ¬x14 ∧ x11 ∧ x22 ∧ x33 ∧ x44):
  (((((¬ x11 ∨ ¬  x12)∧
      (¬ x11 ∨ ¬  x13))∧ 
      (¬ x11 ∨ ¬  x14))∧ 
      (¬ x12 ∨ ¬  x13))∧ 
      (¬ x12 ∨ ¬  x14))∧ 
      (¬ x13 ∨ ¬  x14):=
  show
   (((((¬ x11 ∨ ¬  x12)∧
      (¬ x11 ∨ ¬  x13))∧ 
      (¬ x11 ∨ ¬  x14))∧ 
      (¬ x12 ∨ ¬  x13))∧ 
      (¬ x12 ∨ ¬  x14))∧ 
      (¬ x13 ∨ ¬  x14),
  from and.intro
          (and.intro
                    (and.intro
                          (and.intro
                            (and.intro
                              ( or.inr h2.left ) 
                              ( or.inr h2.right.left ))
                            (or.inr h2.right.right.left))
                          (or.inl h2.left))
                    (or.inl h2.left))
          (or.inl h2.right.left)

--- para a posição 2 ---
lemma rule_4_node2 (h2: ¬x21 ∧ ¬x23 ∧ ¬x24 ∧ x11 ∧ x22 ∧ x33 ∧ x44):
  (((((¬ x21 ∨ ¬  x22)∧
      (¬ x21 ∨ ¬  x23))∧ 
      (¬ x21 ∨ ¬  x24))∧ 
      (¬ x22 ∨ ¬  x23))∧ 
      (¬ x22 ∨ ¬  x24))∧ 
      (¬ x23 ∨ ¬  x24):=
  show
  (((((¬ x21 ∨ ¬  x22)∧
      (¬ x21 ∨ ¬  x23))∧ 
      (¬ x21 ∨ ¬  x24))∧ 
      (¬ x22 ∨ ¬  x23))∧ 
      (¬ x22 ∨ ¬  x24))∧ 
      (¬ x23 ∨ ¬  x24),
  from and.intro
          (and.intro
                    (and.intro
                          (and.intro
                            (and.intro
                              ( or.inl h2.left ) 
                              ( or.inl h2.left ))
                            (or.inl h2.left))
                          (or.inr h2.right.left))
                    (or.inr h2.right.right.left))
          (or.inl h2.right.left)

--- para a posição 3 ---
lemma rule_4_node3 (h2: ¬x31 ∧ ¬x32 ∧ ¬x34 ∧ x11 ∧ x22  ∧ x33 ∧ x44):
  (((((¬ x31 ∨ ¬  x32)∧
      (¬ x31 ∨ ¬  x33))∧ 
      (¬ x31 ∨ ¬  x34))∧ 
      (¬ x32 ∨ ¬  x33))∧ 
      (¬ x32 ∨ ¬  x34))∧ 
      (¬ x33 ∨ ¬  x34):=
  show
  (((((¬ x31 ∨ ¬  x32)∧
      (¬ x31 ∨ ¬  x33))∧ 
      (¬ x31 ∨ ¬  x34))∧ 
      (¬ x32 ∨ ¬  x33))∧ 
      (¬ x32 ∨ ¬  x34))∧ 
      (¬ x33 ∨ ¬  x34),
  from and.intro
          (and.intro
                    (and.intro
                          (and.intro
                            (and.intro
                              ( or.inl h2.left ) 
                              ( or.inl h2.left))
                            (or.inl h2.left))
                          (or.inl h2.right.left))
                    (or.inl h2.right.left))
          (or.inr h2.right.right.left)

--- para a posição 4 ---
lemma rule_4_node4 (h2: ¬x41 ∧ ¬x42 ∧ ¬x43 ∧  x11 ∧ x22  ∧ x33 ∧ x44):
  (((((¬ x41 ∨ ¬  x42)∧
      (¬ x41 ∨ ¬  x43))∧ 
      (¬ x41 ∨ ¬  x44))∧ 
      (¬ x42 ∨ ¬  x43))∧ 
      (¬ x42 ∨ ¬  x44))∧ 
      (¬ x43 ∨ ¬  x44):=
  show
  (((((¬ x41 ∨ ¬  x42)∧
      (¬ x41 ∨ ¬  x43))∧ 
      (¬ x41 ∨ ¬  x44))∧ 
      (¬ x42 ∨ ¬  x43))∧ 
      (¬ x42 ∨ ¬  x44))∧ 
      (¬ x43 ∨ ¬  x44),
  from and.intro
          (and.intro
                    (and.intro
                          (and.intro
                            (and.intro
                              ( or.inl h2.left ) 
                              ( or.inl h2.left))
                            (or.inl h2.left))
                          (or.inl h2.right.left))
                    (or.inl h2.right.left))
          (or.inl h2.right.right.left)

--- REGRA 5 ---

--- para os nós 1 e 3 ---
lemma rule_5_edge13 (h2: ¬x23 ∧ ¬x21 ∧ ¬x43 ∧ ¬x41 ∧  x11 ∧ x22  ∧ x33 ∧ x44):
  (((((¬ x11 ∨ ¬  x23)∧
      (¬ x21 ∨ ¬  x33))∧ 
      (¬ x31 ∨ ¬  x43))∧ 
      (¬ x13 ∨ ¬  x21))∧ 
      (¬ x23 ∨ ¬  x31))∧ 
      (¬ x33 ∨ ¬  x41):=
  show
  (((((¬ x11 ∨ ¬  x23)∧
      (¬ x21 ∨ ¬  x33))∧ 
      (¬ x31 ∨ ¬  x43))∧ 
      (¬ x13 ∨ ¬  x21))∧ 
      (¬ x23 ∨ ¬  x31))∧ 
      (¬ x33 ∨ ¬  x41),
  from and.intro
          (and.intro
                    (and.intro
                          (and.intro
                            (and.intro
                              ( or.inr h2.left ) 
                              ( or.inl h2.right.left))
                            (or.inr h2.right.right.left))
                          (or.inr h2.right.left))
                    (or.inl h2.left))
          (or.inr h2.right.right.right.left)

--- para os nós 3 e 4 ---
lemma rule_5_edge24 (h2: ¬x12 ∧ ¬x24 ∧ ¬x34 ∧ ¬x32 ∧  x11 ∧ x22  ∧ x33 ∧ x44):
  (((((¬ x12 ∨ ¬  x24)∧
      (¬ x22 ∨ ¬  x34))∧ 
      (¬ x32 ∨ ¬  x44))∧ 
      (¬ x14 ∨ ¬  x24))∧ 
      (¬ x24 ∨ ¬  x34))∧ 
      (¬ x34 ∨ ¬  x44):=
  show
  (((((¬ x12 ∨ ¬  x24)∧
      (¬ x22 ∨ ¬  x34))∧ 
      (¬ x32 ∨ ¬  x44))∧ 
      (¬ x14 ∨ ¬  x24))∧ 
      (¬ x24 ∨ ¬  x34))∧ 
      (¬ x34 ∨ ¬  x44),
  from and.intro
          (and.intro
                    (and.intro
                          (and.intro
                            (and.intro
                              ( or.inl h2.left ) 
                              ( or.inr h2.right.right.left))
                            (or.inl h2.right.right.right.left))
                          (or.inr h2.right.left))
                    (or.inl h2.right.left))
          (or.inl h2.right.right.left)
\end{lstlisting}

\section{Exercícios}
