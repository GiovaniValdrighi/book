\newenvironment{bprooftree}
{\leavevmode\hbox\bgroup}
{\DisplayProof\egroup}
  
\chapter{Lógica de Primeira Ordem}
    
    Até agora, vimos Lógica Proposicional, em que variáveis proposicionais podem ser valoradas como verdadeiras ou falsas. Esse sistema lógico, porém, contém limitações. Considere, por exemplo, a afirmação de que todo natural é maior do que ou igual a zero.
    Ou a afirmação de que existe número natural que é primo e é par. Como representar isso em Lógica Propocional? Precisamos de um sistema lógico que saiba lidar com objetos e suas propriedades e relações.

    Talvez esses exemplos não tenham demonstrado a necessidade de tal sistema lógico. Porém, acredite: ao final deste capítulo, essa necessidade ficará evidente.

    \section{Sintaxe}

    A sintaxe de um sistema lógico aborda basicamente os símbolos que são utilizados para representá-lo. Portanto, nesta seção, serão abordados funções, predicados, relações e quantificadores, dentre eles $\forall$ e $\exists$.

    \subsection{Funções, Predicados e Relações}

        Dentro de Lógica de Primeira Ordem, funções, predicados e relações são mapeamentos que, dado algum elemento do domínio, retornam uma proposição ou outro elemento do domínio. Parece confuso? Vamos olhar um exemplo.

        \begin{center}
            $x$ natural é par ou ímpar.            
        \end{center}

        Nosso domínio, neste caso, são os números naturais ($\mathbb{N}$). Dizemos que \textit{par} é ``algo"\space que recebe um número e retorna $V$ ou $F$. Chamamos isso de \textbf{predicado}. Sendo assim, podemos escrever:

        \begin{center}
            $par(x) \lor impar(x)$.
        \end{center}

        A partir deste exemplo, podemos extrair alguns símbolos para exemplificar a construção de um sistema lógico de primeira ordem.

        \begin{itemize}
            \item O domínio são os naturais;
            \item Os objetos são os números $0$, $1$, $2$ etc.;
            \item Existem \textbf{funções}, como \textit{adição} e \textit{subtração}, que recebem (zero ou mais) números e retornam outros números;
            \item Existem \textbf{predicados}, como \textit{par} e \textit{ímpar}, que recebem um número e retornam $V$ ou $F$;
            \item Existem \textbf{relações}, como \textit{igual} e \textit{menor}, que recebem dois números e retornam $V$ ou $F$.
        \end{itemize}

        Os objetos pertencentes ao domínio, chamados constantes, como o $1$ e o $4$, no exemplo, podem ser considerados funções que tomam zero elementos. Além disso, podemos considerar predicados que tomam zero elementos como os valores lógicos $\top$ e $\bot$.

        Expressões que representam elementos do domínio(incluindo funções de elementos) são chamados de \textbf{termos}. Alguns exemplos:

        \begin{itemize}
            \item $5$
            \item $sucessor(10)$ (função sucessora, retorna o elemento acrescido de $1$)
            \item $33+44$
        \end{itemize}

        Observe que o símbolo para a função de adição está ``infixo"; poderíamos ter representado como $+(33, 44)$ ou $adicao(33, 44)$.
        
        Expressões que retornam $V$ ou $F$ são chamadas \textbf{fórmulas}:

        \begin{itemize}
            \item $maiorDoQue(1,2)$ ($1 > 2$)
            \item $par(10) \lor impar(5)$
            \item $2=2$
        \end{itemize}

        O Lean é muito eficiente em expressar Lógica de Primeira Ordem. Vejamos nosso exemplo:

        \begin{lstlisting}
constant U : Type
constant zero : U
constant par : U → Prop
constant primo : U → Prop
constant igual : U → U → Prop
constant adicao : U → U → U
\end{lstlisting}

        Pelo fato de que o Lean é baseado em \textit{Teoria dos Tipos}, declaramos um novo tipo \lstinline{U}. Podemos intuitivamente considerá-lo como ``universo"\space ou ``domínio". Por exemplo, o conjunto dos naturais.

        Foi declarado um objeto chamado \lstinline{zero} do tipo \lstinline{U} (nossa analogia com o zero natural).
        \lstinline{par} é um predicado, pois toma um elemento do tipo \lstinline{U} e retona um elemento do tipo proposição (\lstinline{Prop}).

        \lstinline{adicao} é uma função que toma dois elementos do tipo \lstinline{U} e retorna outro do mesmo tipo. Ora, podemos constatar:

        \begin{lstlisting}
#check par zero
#check adicao zero zero
#check par (adicao zero zero)
\end{lstlisting}

        O \lstinline{#check} da linha $1$ informa que a expressão tem tipo \lstinline{Prop}; na linha $2$, tipo \lstinline{U}; e, na linha $3$, tipo \lstinline{Prop}. Importante observar o papel dos parênteses acima, para que \lstinline{par} receba apenas um elemento.

        Uma função (ou relação) que recebe mais de um elemento tem notação \lstinline{U → U → U}. A notação para predicados (\lstinline{U → Prop}) e relações (\lstinline{U → U → Prop}) funciona como se ambas fossem funções, porém retornassem \lstinline{Prop}. 

        Vários conjuntos estão nas bibliotecas padrão do Lean, como os naturais, utilizados nos exemplos anteriores. O comando para o símbolo \lstinline{ℕ} é \lstinline{\nat} ou \lstinline{\N}.

        \begin{lstlisting}
constant zero : ℕ
#check zero + zero
\end{lstlisting}

        O \lstinline{#check} da linha $2$ retorna algo do tipo \lstinline{ℕ}.

        Podemos misturar Lógica Proposicional com Lógica de Primeira Ordem:

        \begin{lstlisting}
constant U : Type
constant zero : U
constant par : U → Prop
constant primo : U → Prop
constant igual : U → U → Prop
constant adicao : U → U → U

#check ¬ (par zero ∨ par (adicao zero zero)) ∧ primo zero
\end{lstlisting}

        E o \lstinline{#check} nos retorna algo do tipo \lstinline{Prop}.

    \subsection{Quantificador Universal}

        Grande parte do poder da Lógica Proposicional se deve aos quantificadores.
        O símbolo $\forall$ é o quantificador universal, que representa ``para todo".
        Quando ele é seguido de uma variável e de uma expressão, ele indica que aquela expressão é verdadeira para toda variável do domínio.
        Por exemplo:

        \begin{itemize}
            \item $\forall x \ (par(x) \lor impar(x))$
            \item $\forall y \ (par(y) \rightarrow impar(y + 1))$
        \end{itemize}

        A primeira expressão nos diz que todo número é par ou ímpar (no caso dos naturais).
        A segunda diz que, para todo número, o fato dele ser par implica que seu sucessor é ímpar.
        
        No Lean, é possível obter o símbolo \lstinline{∀} digitando \lstinline{\all}. Os exemplos anteriores ficam assim:

        \begin{lstlisting}
constant U : Type
constant par : U → Prop
constant impar : U → Prop
constant sucessor : U → U

#check ∀ x : U, par x ∨ impar x
#check ∀ y : U, par y → impar (sucessor y)
\end{lstlisting}

        Os dois \lstinline{#check}'s nos retornam \lstinline{Prop}. Quando é utilizado o quantificador universal,
        por padrão devemos dizer o tipo da variável que o acompanha. No exemplo, \lstinline{x : U}. Porém, o Lean é
        esperto o suficiente pra inferir o tipo da variável sozinho, então na maioria dos casos não será um problema omiti-lo.

        Observe as três sentenças:

        \begin{itemize}
            \item $\forall x \ (par(x) \lor impar(x))$
            \item $\forall x \ par(x) \lor impar(x)$
            \item $\forall x \ (par(x)) \lor impar(x)$
        \end{itemize}

        Por uma questão de convenção, as duas últimas sentenças são equivalentes, enquanto a primeira é diferente.
        Neste contexto, estamos lidando com o \textbf{escopo} da variável $x$. A convenção diz, portanto, que o escopo da variável é o menor possível.
        
        Curiosamente, o modo como o Lean lida com escopo é diferente: \lstinline{∀ x : U, par x ∨ impar x} equivale a \lstinline{∀ x : U, (par x ∨ impar x)}.
        Ou seja, o Lean busca o maior escopo possível.

        Quando estamos lidando com quantificadores, a variável que o acompanha é dita \textbf{limitada} (\textit{bound}, em inglês).
        Na expressão $\forall x \ A(x)$, a variável $x$ é limitada.
        Isso significa que o $x$ não representa um valor em si, mas apenas um ``espaço reservado"\space para qualquer outra variável.
        Observe que a expressão $\forall y \ A(y)$ representa exatamente a mesma coisa.

        Uma variável que não é limitada é chamada \textbf{livre}. Por exemplo, considere a expressão $\forall x \ y \le x$.
        Ora, sabemos que $x$ representa uma ``espaço reservado"\space para toda variável do domínio. O que representa $y$, portanto?
        Um elemento específico do domínio. Digamos que o domínio é $\mathbb{N}$ e $y$ representa o elemento zero. Então a expressão é verdadeira.
        Sendo assim, se trocarmos $y$ por $z$ (que representa, neste caso, outro elemento), então não vale a expressão $\forall x \ z \le x$.
        Observe como trocar $y$ por $z$ fez toda a diferença!

        No exemplo anterior, tanto $y$ quanto $z$ são variáveis livres. Quando uma expressão \textbf{não} contém variáveis livres, é chamada sentença.

        Quantificadores também possuem regras de introdução e eliminação. Vejamos:
        
        \begin{center}
            \begin{bprooftree}
                \AxiomC{$A(y)$}
                \RightLabel{\scriptsize $\forall I$}
                \UnaryInfC{$\forall x \ A(x)$}
            \end{bprooftree}
            \begin{bprooftree}
                \AxiomC{$\forall x \ A(x)$}
                \RightLabel{\scriptsize $\forall E$}
                \UnaryInfC{$A(t)$} 
            \end{bprooftree}
        \end{center}

        A regra da esquerda demonstra a \textbf{introdução} do quantificador universal. Ela vale quando $y$ não é livre em nenhuma hipótese não cancelada.
        A intuição neste caso é que, dado que vale $A(y)$ para algum $y$ qualquer, então podemos dizer que vale para todo $y$. Mudamos o nome da variável apenas
        pra que a operação fique mais evidente e intuitiva.

        A regra da direita demonstra a \textbf{eliminação} do quantificador universal. A intuição, neste caso,
        é que, se $A(x)$ vale para todo $x$, então vale para algum $t$ qualquer do domínio.

        Observe a semelhança dessas regras com aquelas relativas à implicação. No caso da introdução da implicação, assumimos $A$ e provamos $B$, então $A \rightarrow B$.
        No caso da introdução do quantificador universal, assumimos $x$ e mostramos $A(x)$, então $\forall x \ A(x)$.
        Para a eliminação da implicação, temos $A \rightarrow B$ e $A$, então temos $B$. Já no caso da eliminação do quantificador, temos $\forall x \ A(x)$ e $y$, então $A(y)$.

        Vejamos um exemplo aplicando essas regras. Observe como derivar $\forall x \ A(x) \land B(x)$ a partir de $\forall x \ A(x)$ e $\forall x \ B(x)$:

        \begin{center}
            \begin{bprooftree}
                \AxiomC{$\forall x \ A(x)$}
                \RightLabel{\scriptsize $\forall E$}
                \UnaryInfC{$A(y)$}
                \AxiomC{$\forall x \ B(x)$}
                \RightLabel{\scriptsize $\forall E$}
                \UnaryInfC{$B(y)$}
                \RightLabel{\scriptsize $\land I$}
                \BinaryInfC{$A(y) \land B(y)$}
                \RightLabel{\scriptsize $\forall I  $}
                \UnaryInfC{$\forall x \ A(x) \land B(x)$}
            \end{bprooftree}
        \end{center}

        Ora, e como representar isso em Lean? Vejamos a introdução:

        \begin{lstlisting}
constant U : Type
constant A : U → Prop

example : ∀ x, A x :=
    assume y,
    show A y, from sorry
\end{lstlisting}

        Estamos mostrando $\forall x \ A(x)$ da seguinte forma: assumimos um $y$ qualquer e provamos $A(y)$.
Neste caso, necessitamos de uma prova de $A(y)$, o que justifica o uso do \lstinline{sorry}.

        Observe, agora, a regra da eliminação:

        \begin{lstlisting}
constant U : Type
constant A : U → Prop
constant h1 : ∀ x, A x
constant t : U

example : A t :=
    h1 t
\end{lstlisting}

        Dado que sabemos que $\forall x \ A(x)$ (por \lstinline{h1}), podemos ``aplicar"\space $t$ e obter $A(t)$.

        Vejamos agora nosso exemplo que deriva $\forall x \ A(x) \land B(x)$ a partir de $\forall x \ A(x)$ e $\forall x \ B(x)$:

        \begin{lstlisting}
constant U : Type
constants A B : U → Prop
constant h1 : ∀ x, A x
constant h2 : ∀ x, B x

example : ∀ x, A x ∧ B x :=
    assume y,
        have h3 : A y, from h1 y,
        have h4 : B y, from h2 y,
    show A y ∧ B y, from and.intro h3 h4
\end{lstlisting}

    \subsection{Quantificador Existencial}

        O quantificador existencial é representado pelo símbolo $\exists$ e representa ``existe algum".
        Seguido de uma variável e uma expressão, ele significa que existe algum elemento no domínio tal que a expressão seja verdadeira.
        Alguns exemplos para o domínio dos naturais:

        \begin{itemize}
            \item $\exists x \ x \times x = x$
            \item $\exists y \ y \le 0$
            \item $\forall x \ \exists y \ par(y) \land y > x$
        \end{itemize}

        Ora, as expressões significam:

        \begin{itemize}
            \item Existe número natural que é igual ao seu quadrado ($0$ e $1$)
            \item Existe número menor do que ou igual a zero (o próprio zero)
            \item Para todo natural, existe um número que é par e maior do que ele
        \end{itemize}

        Em Lean, pode-se inserir \lstinline{∃} digitando \lstinline{\ex}. Vejamos os exemplos:

        \begin{lstlisting}
constant U : Type
constant par : U → Prop
constant maiorDoQue : U → U → Prop

#check ∃ x, x * x = x
#check ∃ y, y ≤ 0
#check ∀ x, ∃ y, par y ∧ maiorDoQue y x
\end{lstlisting}

        Como no caso do quantificador universal, a variável que acompanha o quantificador é dita \textbf{limitada}. Por exemplo, $x$ em $\exists x \ A(x)$.
        Variáveis não limitadas são \textbf{livres}.

        Vejamos as regras de introdução e eliminação desse quantificador:
        
        \begin{center}
            \begin{bprooftree}
                \AxiomC{$A(t)$}
                \RightLabel{\scriptsize $\exists I$}
                \UnaryInfC{$\exists x \ A(x)$}
            \end{bprooftree}
            \begin{bprooftree}
                \AxiomC{$\exists x \ A(x)$}
                \AxiomC{}
                \UnaryInfC{$A(y)$}
                \alwaysNoLine
                \UnaryInfC{$\vdots$}
                \UnaryInfC{$B$}
                \alwaysSingleLine
                \RightLabel{\scriptsize $\exists E$}
                \BinaryInfC{$B$}
            \end{bprooftree}
        \end{center}

        A primeira regra é a \textbf{introdução} do quantificador existencial. A intuição que segue é a de que, para mostrar que existe $x$ tal que  $A(x)$, basta mostrar que, para algum $t$, vale $A(t)$.

        A segunda regra é a \textbf{eliminação} e vale quando $y$ não é livre em $B$ e em nenhuma hipótese não cancelada. É uma regra um pouco menos trivial de se entender. Vamos lá:
        
        \begin{itemize}
            \item Digo que existe algum $x$ para o qual vale $A(x)$;
            \item Sei que, independentemente do $y$ que eu escolher, de $A(y)$ é possível derivar $B$;
            \item Concluo que vale B.
        \end{itemize}

        Outra forma de se interpretar essa regra é a seguinte: sei que existe $x$ tal que vale $A(x)$. A única maneira de eu ter certeza de que vale $B$ é mostrando que, pra qualquer $y$, $A(y)$ implica em $B$.
        Tenho que assumir um $y$ qualquer, pois não sei para qual $y$ especificamente vale $A(y)$.

        Tecnicamente, o que estamos fazendo é: dado que vale $\exists x \ A(x)$ e vale $\forall y \ (A(y) \to B)$, concluímos $B$, desde que $B$ não contenha $y$ livre. Para algumas pessoas,
        a utilização do quantificador universal ($\forall$) pode ser muito conveniente. Porém, para outras, pode ser um pouco mais confuso.

        Assim como as regras do quantificador universal se assemelham às da implicação, as regras mostradas acima se assemelham àquelas da introdução e eliminação do operador lógico ou. Observe:

        \begin{itemize}
            \item Assim como de $A$ eu derivo $A \lor B$, de $A(t)$ derivo $\exists x \ A(x)$. O que estamos derivando diz que pelo menos um dos argumentos é válido ou existe.
            \item Para mostrar $C$ de $A \lor B$, assumo $A$ e mostro $C$ e depois assumo $B$ e mostro $C$. No caso do nosso quantificador, para saírmos de $\exists x \ A(x)$ e chegarmos em $B$, assumo qualquer valor de $y$ e mostro que de $A(y)$ chego em $B$.
            Nos dois casos, estamos verificando se, em todas as possibilidades, conseguimos concluir o que queremos.
        \end{itemize}

        Vejamos um exemplo que utiliza as duas regras, mostrando $(\exists x \ A(x)) \land (\exists x \ B(x))$ a partir de $\exists x \ (A(x) \land B(x))$:

        \begin{center}
            \begin{bprooftree}
                \AxiomC{$\exists x \ (A(x) \land B(x))$}
                \AxiomC{}
                \UnaryInfC{$A(y) \land B(y)$}
                \RightLabel{\scriptsize $\land E$}
                \UnaryInfC{$A(y)$}
                \RightLabel{\scriptsize $\exists I$}
                \UnaryInfC{$\exists x \ A(x)$}
                \AxiomC{}
                \UnaryInfC{$A(y) \land B(y)$}
                \RightLabel{\scriptsize $\land E$}
                \UnaryInfC{$B(y)$}
                \RightLabel{\scriptsize $\exists I$}
                \UnaryInfC{$\exists x \ B(x)$}
                \RightLabel{\scriptsize $\land I$}
                \BinaryInfC{$(\exists x \ A(x)) \land (\exists x \ B(x))$}
                \RightLabel{\scriptsize $\exists E$}
                \BinaryInfC{$(\exists x \ A(x)) \land (\exists x \ B(x))$}
            \end{bprooftree}
        \end{center}

        Essas regras também podem ser utilizadas em Lean. Vejamos a introdução:

        \begin{lstlisting}
constant U : Type
constant A : U → Prop
constant t : U
constant h1 : A t

example : ∃ x, A x :=
    exists.intro t h1
\end{lstlisting}

        Para a introdução do quantificador existencial, utilizamos \lstinline{exists.intro}, que requer dois argumentos:
        um termo e uma prova de que a expressão vale para esse termo. É o que está sendo feito no código acima. Na linha $4$, sabemos que vale $A(t)$. Passando $t$ e $A(t)$
        como argumentos para \lstinline{exists.intro}, temos $\exists x \ A(x)$.

        Vejamos a regra da eliminação:

        \begin{lstlisting}
constant U : Type
constant A : U → Prop
constant h1 : ∃ x, A x
constant B : Prop

example : B :=
    exists.elim h1
        (assume y (h2 : A y),
        show B, from sorry)
\end{lstlisting}

        Utilizamos \lstinline{exists.elim}, que toma dois argumentos: a expressão $\exists x \ A(x)$ e uma expressão que prova $B$ a partir de $y$ qualquer e $A(y)$.
        
        Observe como fica mais clara a consideração feita anteriormente com relação ao quantificador universal ($\forall$): estamos dando para \lstinline{exists.elim} as expressões $\exists x \ A(x)$ e $\forall y \ (A(y) \to B)$, e ela nos retorna $B$. Observe como o \lstinline{show} nos ajuda a mostrar isso:

        \begin{lstlisting}
example : B :=
    exists.elim h1
        (show ∀ y, A y → B, from
            assume y (h2 : A y),
            show B, from sorry)
\end{lstlisting}

        Vejamos nosso exemplo anterior, agora implementado em Lean, que mostra $(\exists x \ A(x)) \land (\exists x \ B(x))$ a partir de $\exists x \ (A(x) \land B(x))$:
        \begin{lstlisting}
constant U : Type
constants A B : U → Prop
constant h1 : ∃ x, A x ∧ B x

example : (∃ x, A x) ∧ (∃ y, B y) :=
    exists.elim h1
        (assume y (h2 : A y ∧ B y),
            have h3 : A y, from h2.left,
            have h4 : B y, from h2.right,
            have h5 : ∃ x, A x, from exists.intro y h3,
            have h6 : ∃ x, B x, from exists.intro y h4,
        show (∃ x, A x) ∧ (∃ y, B y), from and.intro h5 h6)
\end{lstlisting}

    \subsection{Relativização}

    Os quantificadores $\forall$ e $\exists$ apresentados até agora variam sobre todos os elementos do domínio.
    Por exemplo, se eu especificar um domínio como sendo as espécies do reino animal, posso dizer que todas são fascinantes da seguinte forma:
    estabeleço um predicado $fascinante(x)$ que retorna $V$ se uma espécie $x$ é fascinante e escrevo: $\forall x \ fascinante(x)$.
    
    Alguém, porém, que goste muito de borboletas, poderia dizer que apenas as borboletas são incríveis. Então ela poderia escrever:
    $\forall x \ (borboleta(x) \rightarrow incrivel(x))$. O que ela está dizendo, então, é: dentre todos os animais, se $x$ é uma borboleta, então $x$ é incrível.
    Como o quantificador universal varia sobre todo o domínio, utilizar a implicação como no exemplo é um truque para restringir o domínio apenas às borboletas.
    Isso é chamado \textbf{relativização}.

    E se, agora, quiséssemos dizer que existe algum cachorro feliz? No caso do quantificador existencial, podemos \textbf{relativizá-lo} utilizando a conjunção:
    $\exists x \ (cachorro(x) \land feliz(x))$. Estamos dizendo que existe algum animal que é um cachorro e é feliz.

    Talvez soe contraintuitivo utilizar a conjunção em detrimento da implicação, como no exemplo anterior. Porém, observe o que acontece se utilizarmos a implicação:
    $\exists x \ (cachorro(x) \rightarrow feliz(x))$. Isso equivale a dizer $\exists x \ (\lnot cachorro(x) \lor feliz(x))$. Ora, estamos dizendo que existe algo que não é cachorro ou é feliz, ou seja, não é bem o que queríamos dizer.

    \subsection{Igualdade}

Na lógica de primeira ordem também é utilizado o símbolo \lstinline{=} para expressar o fato de que duas
expressões se referem ao mesmo objeto, como em "s = t", em que definimos que ambos s e t estão se referindo
ao mesmo objeto do universo. A igualdade é utilizada em expressões como "1 + 1 = 2" ou "seu professor é o
meu pai", note que neste segundo exemplo não utilizamos o símbolo de forma explícita, no entanto, com as 
funções da linguagem afirmamos que "seu professor" e "meu pai" são expressões que se referem ao mesmo objeto.
\newline A definição de igualdade e identidade ao longo da história foram frequentemente debatidas entre
filósofos. Um exemplo é o paradoxo do Navio de Teseu, inspirado na história do herói grego Teseu que era
navegava de Atena para Creta para ser sacrificado em um tributo. No paradoxo, temos A = navio 
que Teseu começou a viagem; e B = navio que Teseu terminou a viagem; ao longo da viagem o navio foi aos poucos
necessitando de reparos, troca de peças, até o momento que ao final da viagem, todas as peças do navio já
haviam sido trocadas, com isso podemos afirmar que A = B? Em uma extensão do paradoxo, existe um outro barco,
o Carniceiro, que pega as peças que Teseu joga ao mar e substitui em seu barco, ao final da viagem o Carniceiro
possui todas as peças que o barco de Teseu tinha ao sair para navegar. Qual dos dois barcos é o de Teseu? Este paradoxo
foi discutido por filósofos como Heráclito, Sócrates, Hobbes, John Locke e Leibniz. No entanto, em lógica ao usarmos
o símbolo de igualdade é pressuposto a nossa interpretação do mundo, e escrever que "s = t" afirma que
s e t representam exatamente o mesmo objeto.
\newline \textbf{Definição: Igualdade} Duas expressões são iguais se referem-se a exatamente o mesmo objeto do universo
e é utilizado o símbolo $=$, chamado de igualdade, para representar essa relação. Possui as
seguintes propriedades:
\begin{itemize}
    \item Reflexão: $t = t$, para qualquer termo $t$.
    \item Simetria: se $s=t$, então $t=s$.
    \item Transitividade: se $s=t$ e $t=v$, então $s=v$.
\end{itemize}
\begin{center}
    \begin{bprooftree}
        \AxiomC{}
        \RightLabel{\scriptsize refl}
        \UnaryInfC{$t=t$}
    \end{bprooftree}
    \begin{bprooftree}
        \AxiomC{$s=t$}
        \RightLabel{\scriptsize sim}
        \UnaryInfC{$t=s$}
    \end{bprooftree}
    \begin{bprooftree}
        \AxiomC{$s=t$}
        \AxiomC{$t=v$}
        \RightLabel{\scriptsize trans}
        \BinaryInfC{$s=v$}
    \end{bprooftree}
\end{center}
\textbf{Exemplo:} No universo $\mathbb{N}$ com constantes $\{ 1, 2,3,5,7\}$ e funções de adição e 
multiplicação, temos duas hipóteses, $2+5 = 3 \cdot 2 +1$ e $7 = 3\cdot 2 +1$. 
Pela propriedade da simetria, $3\cdot 2 +1= 7$, e por fim com a proprieade da transividade podemos 
afirmar que $2+5 = 7$. Representando esta mesma demonstração em uma dedução natural, temos:
\begin{center}
    \begin{bprooftree}
        \AxiomC{$2+5 = 3\cdot 2 +1$}
        \AxiomC{$7 = 3 \cdot 2 + 1$}
        \RightLabel{\scriptsize sim}
        \UnaryInfC{$3 \cdot 2 +1 = 7$}
        \RightLabel{\scriptsize trans}
        \BinaryInfC{$2+7 = 7$}
    \end{bprooftree}
\end{center}
Na dedução, utilizamos apenas a regra de simetria e em sequência a regra de transitividade. Como 
expressariamos esta simples dedução em Lean? A primeira necessidade é definir nosso universo.
\begin{lstlisting}
variable N : Type
variable add : N → N → N
variable mul: N → N → N
variables um dois tres cinco sete: N
\end{lstlisting}]
Definimos a nossa linguagem lógica, com as funções de adição e multiplicação e algumas contantes, números. As 
primeiras propriedades da igualdade em Lean são utilizadas da forma \lstinline{eq.refl} para a propriedade
reflexiva, \lstinline{eq.symm} para a propriedade da simetria e \lstinline{eq.trans} para a propriedade 
da transividade. Neste exemplo iremos apenas utilizar as duas últimas. A dedução termina assim:
\begin{lstlisting}
example (h1 : add dois cinco = add (mul tres dois) um) 
        (h2 : sete =  add (mul tres dois) um):  
(add dois cinco = sete) :=
have h3 : (add (mul tres dois) um = sete), from eq.symm h2,
show add dois cinco = sete, from eq.trans h1 h3
\end{lstlisting}
Na linha 4 utilizamos a primeira regra, a \lstinline{eq.symm} com a nossa hipótese \lstinline{h2}, e terminamos
a prova com a regra \lstinline{eq.trans} utilizando das hipóteses \lstinline{h1} e \lstinline{h3}. Note que 
apesar de ser uma prova com apenas dois passos, foi uma prova extremamente verbosa, isto ocorreu pela 
maneira que definimos a nossa linguagem, para torná-la mais curta podemos utilizar de símbolox infixados
da adição e da multiplicação e poderiamos utilizar os algorimos no lugar do nome dos números, no entanto,
essa segunda sugestão é mais complicada de se realizar. A nova dedudção fica:
\begin{lstlisting}
namespace hidden
constant N : Type
constant add : N → N → N
constant mul: N → N → N
constants um dois tres cinco sete: N
    
infix + := add
infix * := mul
    
example (h1 :  dois + cinco = (tres * dois) + um) 
        (h2 : sete =  (tres * dois) + um): 
        (dois + cinco = sete) :=
    have h3 : (tres * dois) + um = sete, from eq.symm h2,
    show dois + cinco = sete, from eq.trans h1 h3
end hidden 
\end{lstlisting}
A primeira necessidade foi criar um \lstinline{namespace hidden}, pois para definir um \lstinline{infix},
precisamos que a nossa função seja uma \lstinline{costant} e apenas podemos declarar \lstinline{constant}
dentro de um \lstinline{namespace}. Dessa forma, alteramos todas as nossas variáveis para constantes e 
utilizamos na linha 7 e 8 a definição do \lstinline{infix}, a sintaxe é o símbolo que será usado, \lstinline{:=}
e em seguida a função que será infixada.
Essas primeiras propriedades da adição não são suficientes, por exemplo, em uma linguagem com o predicado
par, a função da adição e os números 1 e 2, com as hipóteses $1+1 =2$ e $par(2)$, como provaríamos que 
$par(1+1)$ também vale? É para isso que também são definidas propriedades da subsituição em predicados e funções.
\newline \textbf{Definição:} Seja $s$ e $t$ constantes, $P$ um predicado unário e $r$ uma função unária.
Para a relação de igualdade entre dois objetos também vale as propriedades:
\begin{itemize}
    \item Subsituição em função: se $s=t$, então $r(s) = r(t)$. 
    \item Subsituição em predicado: se $s=t$ e $P(s)$, então $P(t)$.
\end{itemize}
\begin{center}
    \begin{bprooftree}
        \AxiomC{$s=t$}
        \RightLabel{\scriptsize sub}
        \UnaryInfC{$r(s) = r(t)$}
    \end{bprooftree}
    \begin{bprooftree}
        \AxiomC{$s=t$}
        \AxiomC{$P(s)$}
        \RightLabel{\scriptsize sub}
        \BinaryInfC{$P(t)$}
    \end{bprooftree}
\end{center}
\textbf{Exemplo:} Agora com um universo que abrange os números reais, temos o predicado unário $irracional$,
a função unária $raiz$, a função binária $mul$ e as contantes $dois, tres, dezoito$. Com as hipóteses 
$raiz (dezoito) = tres*(raiz(dois))$ e $\forall x, irracional(x \cdot (raiz (dois)))$ queremos provar $irracional(raiz (dezoito))$. O primeiro
passo é a exclusão do universal na segunda hipótese, utilizando a constante $tres$ obtemos $irracional (tres \cdot raiz dois)$,
para aplicar a substituição do predicado, precisamos invertar nossa igualdade utilizando a simetria, e por 
fim utilizamos a regra de substituição. A dedução natural fica:
\begin{center}
    \begin{bprooftree}
        \AxiomC{$raiz(dezoito) = tres \cdot raiz(dois)$}
        \RightLabel{\scriptsize sim}
        \UnaryInfC{$tres \cdot raiz(dois) = raiz(dezoito) $}
        \AxiomC{$\forall x, irracional(x\cdot raiz(dois))$}
        \UnaryInfC{$irracional(tres \cdot raiz(dois))$}
        \RightLabel{\scriptsize subs}
        \BinaryInfC{$irracional(raiz(dezoito))$}
    \end{bprooftree}
\end{center}
A mesma demosntração utilizando o Lean:
\begin{lstlisting}
namespace hidden

constant N : Type
constant mul : N → N → N
constant raiz : N → N
constant irracional : N → Prop
constants dois tres dezoito : N

infix * := mul

example (h1 : raiz dezoito = tres * raiz dois)
        (h2 : ∀ x : N, irracional(x * raiz dois)) :
        irracional (raiz dezoito):=
have h3 : irracional(tres * raiz dois), from h2 tres,
have h4 : tres * raiz dois = raiz dezoito, from eq.symm h1,
show irracional (raiz dezoito), from eq.subst h4 h3

end hidden
\end{lstlisting}
Para a regra de substituição da igualdade utilizamos \lstinline{eq.subst}, como utilizamos na linha 16.
Note que além dessa regra, apenas utilizamos a exclusão do universal utilizando a constante \lstinline{tres}
na linha 14 e a regra de simetria da igualdade na linha 15.
        \subsection{Exercícios}

    \begin{enumerate}
        \item Formalize as afirmações abaixo em Lógica de Primeira Ordem, utilizando os seguintes predicados e relações: $primo(x)$, $par(x)$, $impar(x)$, $perfeito(x)$ e $amigos(x, y)$.
        \begin{enumerate}
            \item Todo número primo maior do que $2$ é ímpar.
            \item A soma de dois números pares é par.
            \item Existe um número perfeito menor do que $10$.
            \item $220$ e $284$ são números amigos.
        \end{enumerate}
        \item Considere que os quantificadores variam sobre estudantes de determinada escola. Formalize as afirmações abaixo, utilizando os seguintes predicados e relações: $amigos(x, y)$, $roqueiro(x)$, $estudioso(x)$, $estudamJuntos(x, y)$ e $irmaos(x, y)$.
        \begin{enumerate}
            \item Roqueiros são amigos de todos.
            \item Existem dois irmãos.
            \item Existe um roqueiro estudioso.
            \item Se duas pessoas estudiosas não estudam juntas, então elas não são irmãs.
        \end{enumerate}
    \end{enumerate}

    \textbf{Soluções:}

    \begin{enumerate}
        \item
        \begin{enumerate}
            \item $\forall x \ ((primo(x) \land x > 2) \rightarrow impar(x))$.
            \item $\forall x \ \forall y \ ((par(x) \land par(y)) \rightarrow par(x + y))$.
            \item $\exists x \ (perfeito(x) \land x < 10)$.
            \item $amigos(220, 284)$.
        \end{enumerate}
        \item
        \begin{enumerate}
            \item $\forall x \ \forall y \ (roqueiro(x) \rightarrow amigos(x, y))$.
            \item $\exists x \ \exists y \ irmaos(x, y)$.
            \item $\exists x \ (roqueiro(x) \land estudioso(x))$.
            \item $\forall x \ \forall y \ ((estudioso(x) \land estudioso(y) \land \lnot estudamJuntos(x, y)) \rightarrow \lnot irmaos(x, y))$.
        \end{enumerate}
    \end{enumerate}
    \section{Semântica}
    
   A seção anterior abordou a sintaxe da lógica de primeira ordem, ou seja, apresentou os símbolos e estruturas das fórmulas deste sistema. A semântica, por outro lado, define a verdade das fórmulas pela \textbf{interpretação} e \textbf{valoração}.
    
    \subsection{Interpretações}
    
    \subsection{Verdade e modelos}
    
    \subsection{Exemplos}
    
    \subsection{Validação e consequência lógica}
    
    \subsection{Correção e completude}
    

    \section{Dedução Natural}
Devemos estabelecer as regras de inclusão e exclusão
de dedução para os quantificadores e para a igualdade. Listamos elas inicialmente:
\newline \textbf{Quantificador universal:}
 \newline 
 \begin{center}
 \begin{bprooftree}
    \AxiomC{$A(y)$}
    \RightLabel{\scriptsize $\forall I$}
    \UnaryInfC{$\forall x A(x)$}
\end{bprooftree}
\begin{bprooftree}
    \AxiomC{$\forall x A(x)$}
    \RightLabel{\scriptsize $\forall E$}
    \UnaryInfC{$A(t)$} 
\end{bprooftree}
\end{center}

Para inserir o quantificador universal devemos ter que a variável $y$ não deve estar 
atrelada em nenhuma hipótese, isto é, em todas as hipóteses não canceladasela não pode ser livre. 
\newline \textbf{Quantificador existencial:}

 \begin{center}
    \begin{bprooftree}
        \AxiomC{$A(t)$}
        \RightLabel{\scriptsize $\exists I$}
        \UnaryInfC{$\exists x A(x)$}
    \end{bprooftree}
    \begin{bprooftree}
        \AxiomC{$\exists x A(x)$}
        \AxiomC{}
        \UnaryInfC{$A(t)$}
        \alwaysNoLine
        \UnaryInfC{$\vdots$}
        \UnaryInfC{$B$}
        \alwaysSingleLine
        \RightLabel{\scriptsize $\exists E$}
        \BinaryInfC{$B$}
    \end{bprooftree}
 \end{center}

Para retirarmmos o quantificador existencial, a variável $t$ não pode estar livre em $B$, isto é, ela não 
deve estar livre em qualquer hipótese não cancelada.

\textbf{Igualdade:}
\begin{center}
    \begin{bprooftree}
        \AxiomC{}
        \RightLabel{\scriptsize refl}
        \UnaryInfC{$t=t$}
    \end{bprooftree}
    \begin{bprooftree}
        \AxiomC{$t=s$}
        \RightLabel{\scriptsize sim}
        \UnaryInfC{$s=t$} 
    \end{bprooftree}
\end{center}
\begin{center}
    \begin{bprooftree}
        \AxiomC{$t=s$}
        \AxiomC{$s=v$}
        \RightLabel{\scriptsize trans}
        \BinaryInfC{$t=v$}  
    \end{bprooftree}
    \begin{bprooftree}
        \AxiomC{$t=s$}
        \RightLabel{\scriptsize subs}
        \UnaryInfC{$r(t) = r(s)$}
    \end{bprooftree}
    \begin{bprooftree}
        \AxiomC{$t=s$}
        \AxiomC{$P(t)$}
        \RightLabel{\scriptsize subs}
        \BinaryInfC{$P(s)$}
    \end{bprooftree}
\end{center}
Iremos passar por cada um destas regras para apresentá-las com o auxílio de exemplos.

\subsection{Quantificador universal}
Como primeiro exemplo segue abaixo uma dedução natural de 
$(\forall x P(x) \to \forall x Q(x)) \to \forall x (P ( x) \to Q (x))$. 
Note que apesar de parecer uma implicação de duas fórmulas idênticas, na premissa a propriedade
 $P$ recebe uma variável e a propriedade $Q$ pode receber outra variável, enquanto que na conclusão
  o valor que $x$ assume é o mesmo para $P$ e $Q$.

\begin{center}
    \AxiomC{}
    \RightLabel{\scriptsize 1}
    \UnaryInfC{$\forall x P(x) \to \forall x Q(x)$}
    \RightLabel{\scriptsize $\forall E$}
    \UnaryInfC{$P(t) \to \forall x Q(x)$}
    \RightLabel{\scriptsize $\forall E$}
    \UnaryInfC{$P(t) \to Q(t)$}
    \RightLabel{\scriptsize $\forall I$}
    \UnaryInfC{$\forall x (P(x) \to Q(x))$}
    \RightLabel{\scriptsize 1}
    \UnaryInfC{$\forall x P(x) \to \forall x Q(x) \to \forall x (P(x) \to Q(x))$}
    \DisplayProof
\end{center}

O primeiro passo será a exclusão da implicação (a implicação principal da fórmula), 
assumindo $\forall x P(x) \to \forall x Q(x)$ como nossa hipótese. Aplicamos uma primeira exclusão do
 universal em $\forall x P(x)$ e novamente aplicamos a exclusão do universal em $\forall x Q(x)$. Note que 
 ao definirmos a variável com o mesmo nome $t$ em ambas as exclusões do universal foi o que permitiu inserir 
 um universal que englobe ambas variáveis, que é o passa seguinte.
\newline Vamos para um outro exemplo, utilizando as hipóteses 
$\forall x (\neg Q(x) \to R(x))$ e $\forall x(P(x) \land \neg Q(x))$ provaremos $\forall x R(x)$:

\begin{center}
    \AxiomC{$\forall x(P(x) \land \neg Q(x))$}
    \RightLabel{\scriptsize $\forall E$}
    \UnaryInfC{$P(t) \land \neg Q(t)$}
    \UnaryInfC{$\neg Q(t)$}
    \AxiomC{$\forall x (\neg Q(x) \to R(x))$}
    \RightLabel{\scriptsize $\forall E$}
    \UnaryInfC{$\neg Q(t) \to R(t)$}
    \BinaryInfC{$R(t)$}
    \RightLabel{\scriptsize $\forall I$}
    \UnaryInfC{$\forall x R(x)$}
    \DisplayProof
\end{center}
Vamos inicialmente utilizar de nossas hipóteses, em $\forall x(P(x) \land \neg Q(x))$ nós executamos a primeira 
exclusão do universal, utilizando a variável $t$ e em seguida executamos a exclusão do $\land$ 
visto que não é necessário o $P(t)$ na dedução. Do outro lado executamos mais uma exclusão do universal
 em $\forall x (\neg Q(x) \to R(x))$ e atribuímos novamente a mesma variável $t$. Com $\neg Q(t)$ e $\neg Q(t) \to R(t)$ 
 podemos realizar uma exclusão da implicaçã e por fim incluímos o universal em $\forall 
 I$.

\subsection{Quantificador existencial}
Lembrando a regra de exclusão do existencial, o que fazemos é que com $\exists x A(x)$, supomos um $y$
arbitrário que satisfaça $A(y)$, a partir desta premissa chegamos até $B$, uma fórmula que não contém
$y$ ou qualquer outra variável aberta em alguma hipótese não cancelada, e podemos concluir $B$.
\newline Já a regra da inclusão do existencial, se uma propriedade vale para um $y$ arbitrário, então
existe uma váriavel em que ela é válida.
\newline Iniciemos com uma demosntração simples de que se existe $x$ que satisfaça $A$ e $B$, então 
existe $x$ que satisfaça $A$.

\begin{center}
    \AxiomC{}
    \RightLabel{\scriptsize 1}
    \UnaryInfC{$\exists x (A(x) \land B(x))$}
    \AxiomC{}
    \RightLabel{\scriptsize 2}
    \UnaryInfC{$A(t) \land B(t)$}
    \UnaryInfC{$A(t)$}
    \UnaryInfC{$\exists x A(x)$}
    \RightLabel{\scriptsize 2}
    \BinaryInfC{$\exists x A(x)$}
    \RightLabel{\scriptsize 1}
    \UnaryInfC{$\exists x (A(x) \land B(x)) \to \exists x A(x)$}
    \DisplayProof
\end{center}
A primeira etapa foi a retirada da implicação, passando $\exists x (A(x) \land B(x))$ como uma
hipótese, e utilizando ela aplicamos a regra de exclusão do existencial, criando uma variável
$t$ arbitrária em que $A(x) \land B(x)$ sejá válido, com isso podemos concluir $A(t)$, note que em $2$
não podemos concluir $A(t)$, pois a variável ainda está aberta em $A(t)\land B(t)$, por isso inserimos
o existencial e concluímos $\exists x A(x)$, resultado que queríamos obter.
\newline O próximo exemplo relaciona os quantificadores universal e existencial, se para todo $x$ $A$
é válido, então existe algum $x$ que $A$ sejá válido.
\begin{center}
    \AxiomC{$\forall x A(x)$}
    \UnaryInfC{$A(t)$}
    \UnaryInfC{$\exists x A(x)$}
    \DisplayProof
\end{center}
Note que se $A(x)$ vale para todo $x$, também vale para um $t$ específico e no passo seguinte poderíamos
tanto utilizar a regra da inclusão do universal ou do existencial. Outro comentário relevante é que não
necessariamente precisávamos concluir o existencial utilizando a mesma variável $x$.
\newline Vamos provar mais uma relação entre os quantificadores, iremos que provar que se para todo $x$ 
não vale $A$, então não existe $x$ tal que $A$ valha:
\begin{center}
    \AxiomC{}
    \RightLabel{\scriptsize 2}
    \UnaryInfC{$\exists x A(x)$}
    \AxiomC{}
    \RightLabel{\scriptsize 3}
    \UnaryInfC{$A(t)$}
    \AxiomC{}
    \RightLabel{\scriptsize 1}
    \UnaryInfC{$\forall x \neg A(x)$}
    \UnaryInfC{$ \neg A(t)$}
    \BinaryInfC{$\bot $}
    \RightLabel{\scriptsize 3}
    \BinaryInfC{$\bot$}
    \RightLabel{\scriptsize 2}
    \UnaryInfC{$\neg \exists x A(x)$}
    \RightLabel{\scriptsize 1}
    \UnaryInfC{$\forall x \neg A(x) \to \neg \exists x A(x)$}
    \DisplayProof
\end{center}
Para a nossa dedução a primeira etapa foi desmontar a implicação, passando $\forall x \neg A(x)$ como
uma de nossas hipóteses, em seguida, como temos uma negação devemos chegar até ao falso. Utilizamos 
$\exists x A(x)$ como mais uma de nossas hipóteses e aplicamos a regra de exclusão do existencial,
dessa forma obtemos no mesmo ramo $A(t)$ e $\neg A(t)$, obtendo a contradição que estávamos procurando.
\subsection{Igualdade}
TO DO 
\subsection{Dedução natural no LEAN}
No Lean a dedução ocorre de forma similar a dedução em primeira ordem, apenas devemos utilizar de novos símbolos e das 
regras de exclusão e inclusão dos quantificadores. Os símbolos $\exists $ e $\forall $ são escritos no Lean como \textbackslash exist 
e  \textbackslash all. Para a regra de exclusão do universal apenas passamos para uma proposição $\forall x A(x)$ uma letra,
por exemplo $t$, para termos $A(t)$. Para a inclusão do universal, devemos assumir uma letra, por exemplo $t$, utilizando o
"assume" e com esta letra livre de qualquer hipótese provamos $A(t)$, dessa forma o Lean é capaz de inferir $\forall x A(x)$.
\newline Vamos utilizando o Lean provar o nosso primeiro exemplo do quantificador universal, $\forall xP(x) \to \forall x Q(x) \to \forall x(P(x) \land Q(x))$:
\begin{lstlisting}
 variable U : Type
 variables P Q : Type $\to$ Prop

 example : ($\forall$x, P x) $\to$ ($\forall$x, Q x) $\to \forall$x P x $\land$ Q x :=  
 assume h$_1$ : $\forall$x, P x,
 assume h$_2$ : $\forall$x, Q x,
 assume t,
 have h$_3$ : P t, from h$_1$ t,
 have h$_4$ : Q t, from h$_2$ t,
 show P t $\land$ Q t, from and.intro h$_3$ h$_4$ 
\end{lstlisting}
Note que na linha 4 utilizamos da inclusão do universal, assumimos um $t$ e desejamos provar $P(t) \land Q(t)$ 
sem que $t$ possua qualquer restrição e nas linhas 5 e 6 utilizamos a exclusão do universal, temos fórmulas do
tipo $\forall x P(x)$ e passamos a letra $t$, obtendo $P(t)$. Além disso note nesse exemplo a importância da utilização
dos parenteses para delimitar a ação do quantificador, quando definimos o exemplo, caso não tivessemos colocado
o parenteses em  $(\forall x, P x)$, o primeiro quantificador $\forall x$ seria interpretado como se valesse para toda
a expressão, incluindo as duas implicações.
\newline Vamos também provar o segundo exemplo em Lean:
\begin{lstlisting}
variable U : Type
variables P Q R : U $\to$ Prop

example (h$_1$ : $\forall$ x, P x $\land \neg$ Q x) (h$_2$ : $\forall$ x, $\neg$ Q x $\to$ R x) :
$\forall$ x, R x :=
assume t,
have h$_3$ : P t $\land$ $\neg$ Q t, from h$_1$ t,
have h$_4$ : $\neg$ Q t $\to$ R t, from h$_2$ t,
show R t, from h$_4$ h$_3$.right
\end{lstlisting}
Note que neste simples exemplo, apesar de possuirmos fórmulas maiores, apenas realizamos simples regras de inclusão e exclusão
do universal.
\newline Para as regras do existencial, na exclusão utilizamos "exists.elim" seguida por uma proposição do tipo $\forall x A(x)$,
para provarmos $B$, devemos provar $ A y \to B$, ou seja, assumimos um $y$ e $A y$ e chegamos até $B$, concluindo assim a exclusão
do existencial. Para a inclusão do existencial utilizamos "exists.intro", devemos passar uma letra, por exemplo $t$, e uma prova
de que $A(t)$ vale.
\newline Vamos provar o nosso primeiro exemplo existencial com o Lean, $\exists x (A(x) \land B(x)) \to \exists x A(x)$:
\begin{lstlisting}
variable U : Type
variables A B : U $\to$ Prop

example : ($\exists$x, A x $\land$ B x) $\to \exists$x, A x :=
assume h$_1$ : $\exists$ x, A x $\land$ B x,
exists.elim h$_1$
    (assume t (h$_2$ : A t $\land$ B t)
        show $\exists$ x, A x, from exists.intro t h$_2$.left) 
\end{lstlisting}
\subsection{Exercícios}
\textbf{1.} Temos que 
que $A \to A \lor B$, apresente uma dedução natural 
para $(\forall x A (x)) \to (\forall x A(x)\lor B(x))$. Em seguida,
demonstre utilizando Lean.
\newline \textbf{2.} Assim como temos em lógica proposicional
 que $A \land B \to A$, de uma dedução natural para 
 $(\forall x A (x) \land B(x)) \to (\forall x A(x))$. Em seguinda,
 demonstre utilizando Lean.
\newline \textbf{3.} Apresente uma dedução natural e uma
prova Lean para a relação $(\forall x A(x) \to \neg B(x)) \iff \neg (\exists x A(x) \land B(x))$.
 \newline \textbf{4.}Resolve as seguintes deduções naturais com auxílio do Lean.
\newline \textbf{a)} $\forall x A(x) \iff \neg \exists x \neg A(x)$.
\newline \textbf{b)} $\forall x(A(x) \to B(x)) \to (\forall x A(x) \to \forall x B(x))$.
\newline \textbf{c)} Utilizando $\forall y \neg A (y)$ e $\forall x A(x)
\lor B(x)$ prove $\forall x B(x)$.
\newline \textbf{d)} Utilizando $\forall x A(x) \land B(x)$ e 
$\forall x A(x) \land B(x) \to C(x)$ prove $\forall x A(x) \land C(x)$.
\newline \textbf{e)}De uma prova de $\forall y P(y) \to P(f (f y))$ utilizando 
de $\forall x P(x) \to P(f(x))$.
\newline \textbf{5.} Utilizando dos predicados unários 
$even$ e $odd$, e da função unária $succ$, das hipóteses
$\forall x even(x) \lor odd(x)$ e $\forall x odd(x) \to even(succ(x))$,
prove que $\forall x even(x) \lor even(succ(x))$.
\newline \textbf{6.} Prove que $y = x → y = z → x = z$.
\newline \textbf{7.} Retornando ao capítulo anterior com o exercício de Ana, Cláudia e Maria e seus três vestidos
agora podemos resolver o problema utilizando de proposição de primeira ordem. Na situação, temos: 
\newline Três irmãs - Ana, Maria e Cláudia - foram a uma festa com vestidos de
cores diferentes. Uma vestia azul, a outra branco e a Terceira
preto. Chegando à festa, o anfitrião perguntou quem era cada uma
delas. As respostas foram:
\newline - A de azul respondeu: “Ana é a que está de branco”
\newline - A de branco falou: “Eu sou Maria”
\newline - A de preto disse:  “Cláudia é quem está de branco”
\newline O anfitrião foi capaz de identificar corretamente quem era cada pessoa
considerando que:
\newline - Ana sempre diz a verdade
\newline - Maria às vezes diz a verdade
\newline - Cláudia nunca diz a verdade
\newline Pensando um pouco sobre o problema, pode-se concluir que a Ana estava
com o vestido preto, a Cláudia com o branco e a Maria com o
azul.  
\newline \textbf{a)} Escreva as fórmulas de primeira ordem necessárias para o problema.
\newline \textbf{b)} Utilizando da ferramente Lean, apresente uma
deduração natural que prove que Ana veste preto, Cláudia
branco e Maria azul.
\subsection{Gabarito}
\textbf{1.}
\begin{lstlisting}
variable U : Type
variables A B : U → Prop
--term mode
example : (∀ x, A x) → ∀ x, A x ∨ B x :=
assume h1: ∀ x, A x,
assume t,
have h2 : A t, from h1 t,
or.inl h2

--tatics
example : (∀ x, A x) → ∀ x, A x ∨ B x :=
begin
intros h1 t,
exact or.inl (h1 t)
end
\end{lstlisting}
\textbf{2.}
\begin{lstlisting}
variable U : Type
variables A B : U → Prop
--term mode
example : (∃ x, A x ∧ B x) → ∃ x, A x :=
assume h1 : ∃ x, A x ∧ B x,
exists.elim h1
    (assume (t) (h2 : A t ∧ B t),
    have h3 : A t, from h2.left,
    show ∃ x, A x, from exists.intro t h3)
    
--tatics
example : (∃ x, A x ∧ B x) → ∃ x, A x :=
begin
intros h1,
cases h1 with t h2,
apply exists.intro t,
exact h2.left
end    
\end{lstlisting}
\textbf{3.}
\begin{lstlisting}
variable U : Type
variables A B : U → Prop 

--term mode
example : (∀ x, A x → ¬ B x) ↔ ¬ ∃ x, A x ∧ B x :=
iff.intro
  (assume h1 : ∀ x, A x → ¬ B x,
  assume h2 : ∃ x, A x ∧ B x,
  exists.elim h2
    (assume (t) (h3 : A t ∧ B t), 
    have h4 : A t → ¬ B t, from h1 t,
    have h5 : ¬ B t, from h4 h3.left,
    show false, from h5 h3.right))
  (assume h2 : ¬ ∃ x, A x ∧ B x,
  assume t, 
  assume h3 : A t,
  assume h4 : B t,
  have h5 : A t ∧ B t, from and.intro h3 h4,
  have h6 : ∃ x, A x ∧ B x, from exists.intro t h5,
  show false, from h2 h6)

--observe que no term mode podemos passar mais de uma hipótse
--para o comando assume
example : (∀ x, A x → ¬ B x) ↔ ¬ ∃ x, A x ∧ B x :=
iff.intro
  (assume (h1 : ∀ x, A x → ¬ B x) (h2 : ∃ x, A x ∧ B x),
  exists.elim h2
    (assume (t) (h3 : A t ∧ B t), 
    have h4 : A t → ¬ B t, from h1 t,
    have h5 : ¬ B t, from h4 h3.left,
    show false, from h5 h3.right))
  (assume (h2 : ¬ ∃ x, A x ∧ B x) t (h3 : A t) (h4 : B t),
   have h5 : A t ∧ B t, from and.intro h3 h4,
   have h6 : ∃ x, A x ∧ B x, from exists.intro t h5,
   show false, from h2 h6)  

--tatics
example : (∀ x, A x → ¬ B x) ↔ ¬ ∃ x, A x ∧ B x :=
begin
apply iff.intro,
intros h1 h2,
cases h2 with t h3,
exact ((h1 t) h3.left) h3.right,
intros h1 t h2 h3,
exact h1 (exists.intro t (and.intro h2 h3))
end

\end{lstlisting}
\textbf{4)a):}

\begin{lstlisting}
variable U : Type
variables A : U → Prop
    
--ida  
--term mode
example : (∀ x, A x) → (¬ ∃ x, ¬ A x) :=
assume h1 : ∀ x, A x,
assume h2 : ∃ x, ¬ A x, show false, from
    (exists.elim h2
        (assume t (h3 : ¬ P t), 
            have h4 : A t, from h1 t, h3 h4))
    
--tatics
example : (∀ x, A x) → (¬ ∃ x, ¬ A x) :=
begin
intros h1,
intro h2,
apply exists.elim h2,
intro t,
intro h3,
exact h3 (h1 t)
end 

--volta
--term mode
example : (∃ x, ¬ A x) → ¬ ∀ x, A x :=
assume h1 : ∃ x, ¬ A x,
assume h2 : ∀ x, A x, 
show false, from 
    (exists.elim h1
        (assume t (h3 : ¬ A t),
            have h4 : A t, from h2 t,
            h3 h4))
        
--tatics

example : (∃ x, ¬ A x) → ¬ ∀ x, A x :=
begin
intro h1,
intro h2,
apply exists.elim h1,
intro t,
intro h3,
exact h3 (h2 t)
end
\end{lstlisting}
\textbf{4.b)}
\begin{lstlisting}
variable U : Type
variables A B : U → Prop 

--term mode
example : (∀ x, A x → B x) → ((∀ x, A x) → ∀ x, B x) :=
assume h1 : ∀ x, A x → B x, 
assume h2 : ∀ x, A x, 
assume t, 
have h3 : A t → B t, from h1 t,
have h4 :  A t, from h2 t,
show B t, from h3 h4


--tatics
example : (∀ x, A x → B x) → ((∀ x, A x) → ∀ x, B x) :=
begin
intros h1 h2 t,
exact (h1 t) (h2 t)
end
\end{lstlisting}
\textbf{4.c)}
\begin{lstlisting}
open classical
variable U : Type
variables A B : U → Prop
--term mode
example (h1 : ∀ y, ¬ A y) (h2: ∀ x, A x ∨ B x) : ∀ x, B x :=
assume t, 
have h3 : ¬ A t, from h1 t,
have h4 : A t ∨ B t, from h2 t,
or.elim h4
    (assume h5 : A t, 
    by_contradiction
    (assume h6 : ¬ B t, show false, from h3 h5))
    (assume h6 : B t, show B t, from h6)

--tatics
example (h1 : ∀ y, ¬ A y) (h2: ∀ x, A x ∨ B x) : ∀ x, B x :=
begin
intro t,
cases (h2 t),
apply classical.by_contradiction,
intro h3,
exact (h1 t) h,
exact h
end   
\end{lstlisting}
\textbf{4.d)}
\begin{lstlisting}
variable U : Type
variables A B C : U → Prop 
    
--term mode
example (h1 : ∃ x, A x ∧ B x) (h2 : ∀ x, (A x ∧ B x → C x)) : ∃ x, A x ∧ C x :=
exists.elim h1 
    (assume t (h3 : A t ∧ B t),
    have h4 : A t ∧ B t → C t, from h2 t,
    have h5 : C t, from h4 h3,
    have h6 : A t ∧ C t, from and.intro  (h3.left) h5,
    show ∃ x, A x ∧ C x, from exists.intro t h6)
    
--tatics
example (h1 : ∃ x, A x ∧ B x) (h2 : ∀ x, (A x ∧ B x → C x)) : ∃ x, A x ∧ C x :=
begin
cases h1 with t h3,
apply exists.intro t,
apply and.intro,
exact h3.left,
exact (h2 t) h3
end  
\end{lstlisting}
\textbf{4.e)}
\begin{lstlisting}
variable U : Type
variable f : U → U
variable P : U → Prop
example  (h1 : ∀ x, P x → P (f x)) : ∀ y, P y → P ( f (f y)) :=
begin
intros t h2,
exact (h1 (f t))((h1 t) h2)
end  
\end{lstlisting}
\textbf{5.}
\begin{lstlisting}
variables even odd: U → Prop
variable s : U → U
example (h1 : ∀ x, even(x) ∨ odd(x)) (h2: ∀ x, odd(x) → even(s(x))) 
: (∀ x, even(x) ∨ even(s(x))) :=
assume t,
have h3 : even(t) ∨ odd(t), from h1 t,
have h4 : odd(t) → even(s(t)), from h2 t,
or.elim h3
    (assume h5 : even(t), show even(t) ∨ even(s(t)), 
    from or.inl h5)
    (assume h5 : odd(t),
    have h6: even(s(t)), from h4 h5,
    show even(t) ∨ even(s(t)), from or.inr h6)
    
example (h1 : ∀ x, even(x) ∨ odd(x)) (h2: ∀ x, odd(x) → even(s(x))) 
: (∀ x, even(x) ∨ even(s(x))) :=
begin
intro t,
apply or.elim (h1 t),
intro h3,
apply or.inl,
exact h3,
intro h3,
apply or.inr,
exact (h2 t) h3
end  
\end{lstlisting}
\textbf{6.}
\begin{lstlisting}
variable N : Type
variables x y z : N
    
example : y = x → y = z → x = z :=
assume h1 : y = x,
assume h2 : y = z,
show x = z, from eq.trans (eq.symm h1) h2
    
example : y = x → y = z → x = z :=
begin
intros h1 h2,
apply eq.trans,
apply eq.symm,
exact h1,
exact h2
end
\end{lstlisting}
\textbf{7.}
\begin{lstlisting}
section 
variable {pessoa : Type} 
variable {cor : Type}
variables {Ana Maria Claudia : pessoa}
variables {preto branco azul : cor}
variable {veste : pessoa → cor → Prop}

-- Restricoes:

-- Se uma usa uma cor, as outras nao podem usar essa cor
variable r1 : ∀c, ∀x, ∀y, (veste x c) ∧ ¬(x = y) → ¬(veste y c)
-- Se uma pessoa usa tal cor, nao pode usar as outras cores
variable r2 : ∀x, ∀c, ∀d, (veste x c) ∧ ¬(c = d) → ¬(veste x d)

-- Raciocinios basicos:

--Se Ana for a de azul, ela esta de branco
variable h1 : (veste Ana azul) → (veste Ana branco)
--Se Ana for a de branco, entao Maria esta de branco
variable h2 : (veste Ana branco) → (veste Maria branco)
--Se ana for de preto, entao Claudia esta de branco
variable h3 : (veste Ana preto) → (veste Claudia branco)
--Se Ana não usa azul e nem branco, ela esta de preto
variable h4 : ¬ (veste Ana azul) ∧ ¬ (veste Ana branco)  →  (veste Ana preto)
--Se Ana veste preto e Claudia veste branco, entao Maria esta de azul
variable h5 : (veste Ana preto) ∧ (veste Claudia branco) →  (veste Maria azul)

variable d1 : ¬ (azul = branco)
variable d2 : ¬ (Ana = Maria)
-- prova de que Ana veste preto
lemma Ana_p : (veste Ana preto) :=
have n1 : ¬ (veste Ana azul), from 
assume m1 : veste Ana azul, show false, from 
    have q1 : veste Ana branco, from h1 m1,
    have q2 : ¬ veste Ana branco, from (r2 Ana azul branco (
        and.intro m1 d1)),
q2 q1,
have n2 : ¬ (veste Ana branco), from
assume m2 : veste Ana branco, show false, from
    have q1 : veste Maria branco, from h2 m2,
    have q2 : ¬ veste Maria branco, from (r1 branco Ana Maria (and.intro m2 d2)),
q2 q1,
show veste Ana preto, from h4 (and.intro n1 n2)

theorem solution : (veste Ana preto) ∧ (veste Claudia branco) ∧ (veste Maria azul) :=
have n1 : veste Ana preto, from (Ana_p r1 r2 h1 h2 h4 d1 d2),
have n2 : veste Claudia branco, from h3 (Ana_p r1 r2 h1 h2 h4 d1 d2),
have n3 : veste Maria azul, from h5 (and.intro n1 n2),
and.intro n1 (and.intro n2 n3)

end
\end{lstlisting}